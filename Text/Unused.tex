As outlined in section \ref{s_int_doppler_shift_method}, the three principal steps in determining a user position from Doppler frequency measurements of a satellite signal are: determining the identity and position (or trajectory) of the transmitting satellite, acquiring and tracking the transmitted signal and measuring its frequency, calculating the user position from the measured data and compensating errors. 

These three steps can be further broken down to:
\begin{enumerate}
    \item identifying the transmitting satellite,
    \item determining the position or trajectory of the transmitting satellite,
    \item acquiring the transmitted signal,
    \item tracking the transmitted signal,
    \item measuring the frequency of the transmitted signal,
    \item calculating the user position from the measured data,
    \item estimating and compensating the errors introduced in each of the previous steps.
\end{enumerate}

\section{The Doppler curve fitting method algorithm}
The principle of the Doppler curve fitting method is described in section \ref{s_transit_nav_method}. In summary, a Doppler shift curve is obtained, then the zero-Doppler-shift time is estimated, and an array of possible (trial) user locations is created based on this time. For each location, a trial Doppler curve is created and then compared against the measured curve, until a sufficiently fitting curve is found.

The method works over observations collected as a set of points, each containing time of reception ($t_R$), received frequency ($f_R$), estimated carrier frequency ($f_B$), an estimation of satellite position and velocity at the time of reception from the SGP4 model ($R_S$, $V_S$), and the satellite ID (\texttt{sat-id}).


\subsection{Curve extraction}
First Doppler curves are extracted from the observation. A Doppler curve is defined as a curve of Doppler-shifted frequency measurements in time, belonging to one channel of one transmitting satellite, sorted by time. That is, a curve if formed by all observations with constant $f_B$ and satellite ID. Furthermore, for a correct computation, the curve must be "nice" - its observations must be mostly equally distributed (variance criterion), sufficiently numerous (density criterion), sufficiently apart in time (duration criterion), without significant gaps (gap criterion) and the $f_R$ must pass through the $f_B$ of the curve (trend criterion).

The extraction of the curves follows a simple rule: an observation $m_i$ belongs to a curve $M_j$ if the respective $f_B$ and \texttt{sat-id} are equal:

\begin{equation*}
    m_i(t_R, f_R, f_B, R_S, V_S, \texttt{sat-id}) \in M_j(f_B, \texttt{sat-id}) \Leftrightarrow f_{B_i}, \texttt{sat-id}_i = f_{B_j}, \texttt{sat-id}_j
\end{equation*}

The curve $M_j$ containing $n$ observations $m_i$ ($i \in <0, n>$) is a valid curve for further computation if all conditions listed below are satisfied:

\begin{align*}
    \text{Duration}&: t_{R_n} - t_{R_0} \geq \texttt{MIN-TIME} \\
    \text{Variance}&: \sum^n_{i=1} |f_{R_{i-1}} - f_{R_i}| \leq \texttt{MAX-VARIANCE} \\
    \text{Density}&: \frac{n}{t_{R_n} - t_{R_0}} \geq \texttt{MIN-DENSITY} \\
    \text{Gap}&: \max_{i \in <1, n>}(|t_{R_{i-1}} - t_{R_i}|) \leq \texttt{MAX-GAP} \\
    \text{Trend} &: f_{R_0} > f_B > f_{R_n} \lor f_{R_0} < f_B < f_{R_n}
\end{align*}
where \texttt{MIN-TIME}, \texttt{MAX-VARIANCE}, \texttt{MIN-DENSITY}, and \texttt{MAX-GAP} are parameters. Note that for the trend criterion, strict criterion of $f_{R_{i-1}} \leq f_{R_i}$ or $f_{R_{i-1}} \geq f_{R_i}$ cannot be applied due to measurement noise.


\subsection{Zero-Doppler-shift time}
The moment at which the Doppler shift is zero is the moment at which the satellite is stationary relative to the user. Simply, it is the time $t_Z$ when $f_R = f_B$. It is improbable this moment will be captured, therefore it is necessary to interpolate. Specifically, the interpolated function is $f_d = f(t)$, where $f_d = f_R - f_B$, interpolated at $f_d = 0$. The function \texttt{numpy.interp} was used.

After calculation of $t_Z$, a satellite position $R_Z$ and velocity $V_Z$ at this time can be simulated using the SGP4 model. The position is then transformed into geodetic frame of reference to acquire the satellite ground location (latitude and longitude) $P_Z$.

\subsection{Trial location generation}
The user is located somewhere on a plane perpendicular to the satellite velocity $V_Z$ and passing through the point $R_Z$. This plane is too large to be searched iteratively, therefore, it needs to be constrained.

The first constraint is altitude - \texttt{MAX-ALTITUDE}. The user can be assumed to be above ground level. For simple applications, the user can also be assumed to be below \qty{2}{km} in altitude, or the altitude can omitted completely\footnote{As was done in the Transit system}. The second constraint is maximum distance from the satellite, given as the maximum line-of-sight distance:

\begin{figure}
    \centering
    \includegraphics[width=\linewidth]{img/des_early_result.png}
    \caption{Enter Caption}
    \label{f_des_early_result}
\end{figure}

\begin{figure}
    \centering
    \includegraphics[width=\linewidth]{img/des_symmetry.png}
    \caption{Enter Caption}
    \label{f_des_symmetry}
\end{figure}
