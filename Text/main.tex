% arara: pdflatex: { synctex: yes }
% arara: makeindex: { style: ctuthesis }
% arara: bibtex

% The class takes all the key=value arguments that \ctusetup does,
% and a couple more: draft and oneside
\documentclass[twoside]{style/ctuthesis}

\ctusetup{
	% preprint = \ctuverlog,
	mainlanguage = english,
	titlelanguage = english,
	otherlanguages = {czech},
	title-english = {Signals of opportunity positioning using LEO satellites},
    title-czech = {Využití signálů družic na nízké oběžné dráze k určování polohy},
	doctype = M,
	faculty = F3,
	department-english = {Department of Measurement},
	author = {Bc. Vojtěch Voska},
	supervisor = {Ing. Jiří Svatoň, Ph.D.},
	fieldofstudy-english = {Aerospace Engineering},
	subfieldofstudy-english = {Avionics},
	keywords-czech = {navigace příležitostnými signály, Dopplerovská navigace, Dopplerův jev, Iridium, Orbcomm, Globalstar, Starlink, NAVSAT, Argos, GNSS, fázové klíčování, satelitní konstelace, LEO satelity, softwarově definované rádio},
	keywords-english = {Signals Of Opportunity Navigation, Doppler Positioning, Doppler Shift, Iridium, Orbcomm, Globalstar, Starlink, NAVSAT, Argos, GNSS, Phase-Shift Keying Modulation, Satellite Constellation, LEO Satellites, Software-Defined Radio},
	day = 23,
	month = 5,
	year = 2024,
	specification-file = {Assignment.pdf},
    front-specification = true,
	front-list-of-figures = true,
	front-list-of-tables = true,
%	monochrome = true,
}

\ctuprocess

\addto\ctucaptionsczech{%
	\def\supervisorname{Vedoucí}%
	\def\subfieldofstudyname{Studijní program}%
}

\ctutemplateset{maketitle twocolumn default}{
	\begin{twocolumnfrontmatterpage}
		\ctutemplate{twocolumn.thanks}
		\ctutemplate{twocolumn.declaration}
		\ctutemplate{twocolumn.abstract.in.titlelanguage}
		\ctutemplate{twocolumn.abstract.in.secondlanguage}
		\ctutemplate{twocolumn.tableofcontents}
		\ctutemplate{twocolumn.listoffigures}
	\end{twocolumnfrontmatterpage}
}

% Abstract in Czech
\begin{abstract-czech}
Tato práce je studií a experimentální demonstrací systému určení polohy na základě zpracování oportunních signálů Dopplerovskou metodou. Tato metoda určení polohy využívá příjmu signálů vyslaných za jiným účelem než navigace a měření jejich Dopplerovské frekvence. Dopplerovská metoda je popsána a některé satelitní systémy na nízké orbitě Země potenciálně použitelné pro oportunní navigaci jsou diskutovány. Je provedena rešerše soudobého výzkumu v této oblasti, a vyvinuté systémy jsou porovnány s dedikovanými satelitními navigačními systémy. Je detailně popsán vývoj plnohodnotného oportunního navigační systému, schopného přijmout a zpracovat signál, spočítat polohu vysílacího satelitu a odhadnou polohu statického uživatele. Tento systém je pak experimentálně prověřen a jeho parametry analyzovány.
\end{abstract-czech}

% Abstract in English
\begin{abstract-english}
This work is a study and an experimental demonstration of satellite-aided Signals of Opportunity (SoP) positioning using the Doppler shift method. This method of positioning uses the reception of satellite signals which were transmitted for other purposes than navigation to determine the position of the user by means of measuring the Doppler shift of the signal. The Doppler method is discussed and some Low Earth Orbit satellite communication systems potentially useful for SoP positioning are described. A survey of current research on the topic is conducted and the SoP systems developed therein are compared to dedicated navigation systems. A full-stack SoP navigation system, capable of signal capture, satellite position determination and static user position estimation is developed and described in detail. The system is then experimentally evaluated and its parameters analysed.
\end{abstract-english}

% Acknowledgements / Podekovani
\begin{thanks}
I would like to thank my family for all their support, patience and understanding.

I would also like to thank my friends for their advice during the times I felt stuck, and for their help in the tedious task of proofreading this thesis.

Last but not least, I would like to thank my supervisor for his support, advice, patience and quick replies.
\end{thanks}

% Declaration / Prohlaseni
\begin{declaration}
I, the author, hereby declare that I have completed this thesis independently without illegitimate assistance, and that I have stated all used literature in accordance with the Methodical Instruction no. 1/2009, concerning ethical principles of authoring a university thesis at Czech Technical University in Prague.

\vspace{10pt}

In Prague, \ctufield{day}~\monthinlanguage{title}~\ctufield{year} \hfill \ldots\ldots\ldots

\vspace{10pt}

This thesis uses several Free and Open Source programs. Attributions per the respective program's license are within the thesis text, where appropriate, and all are listed in Appendix.
\end{declaration}

\sisetup{
    range-units = single,
    % list-units = single,
    uncertainty-mode = separate
}

% custom commands
\newcommand{\citep}[2]{\cite[p. #2]{#1}}
\newcommand{\param}[1]{\texttt{\MakeUppercase{#1}}}
\newcommand{\variation}[1]{\paragraph{#1}}
\newcommand{\file}[1]{\paragraph{\texttt{#1}}}
\renewcommand{\sc}[1]{\textsuperscript{#1}}

\begin{document}
\maketitle

\chapter{Introduction}

\section{Advantages of SoP positioning}

\section{Possible use cases}

\chapter{Doppler frequency-based satellite-aided positioning}
\label{s_pos}
This chapter lays the groundwork for the system designed in the latter part of this work. The Doppler effect and the Doppler frequency-based method, henceforth referred to as the \textit{Doppler method}, are described, with some implementations based on the reviewed literature. Furthermore, this chapter discusses the process of satellite signal capture, which is an essential part of satellite-aided navigation. A way to determine satellite position is presented, and frames of reference used for navigation in this work are described. 



\section{The Doppler effect}
The Doppler effect is the change in the frequency of a wave as perceived by an observer who is moving relative to the source of the wave. The received frequency can be expressed as
\begin{equation}
    \label{e_pos_doppler_basic}
    f_R = \frac{c \pm v_R}{c \pm v_T} f_0
\end{equation}
 where $c$ is the speed of the wave in the medium, $v_R$ and $v_T$ are the respective scalar velocities of the receiver and transmitter relative to the medium (the direction depends on the sign convention), and $f_0$ is the transmitted frequency. 

To simplify calculations, the \autoref{e_pos_doppler_basic} can be simplified:

\begin{align*}
    f_R &= \frac{c + v_R}{c + v_T} f_0 \\
    \text{dividing by } c: f_R & = (\frac{1 + \frac{v_R}{c}}{1 + \frac{v_T}{c}}) f_0 \\
\end{align*}
Assuming $c \gg v_T$ and therefore $\frac{v_T}{c} \ll 1$, a substitution using a truncated Taylor series expansion $\frac{1}{1 + x} \approx (1 - x)$ can be made:

\begin{align}
    f_R &= \left( \frac{1}{1 + \frac{v_R}{c}} \right) \times \left( \frac{1}{1 - \frac{v_T}{c}} \right) f_0 \nonumber \\
    f_R &= \left( 1 - \frac{v_T}{c} + \frac{v_R}{c} - \frac{v_R v_T}{c^2} \right) f_0 \nonumber \\
    \frac{v_R v_T}{c^2} \rightarrow 0: f_R &= \left( 1 + \frac{v_R - v_T}{c} \right) f_0 \nonumber \\
    \text{substituting } \Delta v = v_R - v_T: f_R &= \left( 1 + \frac{\Delta v}{c} \right) f_0 \label{e_pos_dopp_fr}
\end{align}
% Kde jste tahle odvození vzal :-) ? To bych asi jen tak z hlavy %nedal. Hlavně pokud jsou ok. (příp. citace.). Ale výsledek je %jasný. OK.
%% popravdě z Wikipedie. Výsledná rovnice se hojně objevuje v literatuře, ale ne to odvození.
$\Delta v$ is the relative velocity of the receiver and transmitter, alternatively labelled as range rate $\dot\rho$. By convention it is positive if the range between the transmitter and receiver decreases.

The Doppler shift $f_D = f_R - f_0$ can be expressed from \autoref{e_pos_dopp_fr} as:

\begin{equation}
\label{e_pos_dopp_fd}
f_D = \frac{\Delta v}{c} f_0 
\end{equation}

In vector form, the range rate $\dot\rho$ can be expressed as the projection of the relative velocity vector $\Vec{v_T} - \Vec{v_R}$ on the relative position vector $\Vec{r_T} - \Vec{r_R}$:

\begin{equation}
    \label{e_pos_range_rate}
    \dot\rho_i = (\Vec{V_T} - \Vec{V_R}) \frac{\Vec{r_T} - \Vec{r_R}}{||\Vec{r_T} - \Vec{r_R}||}
\end{equation}

Thus, for a stationary receiver ($\Vec{v_R} = \Vec{o}$), the Doppler frequency can be expressed as
\begin{equation}
    \label{e_pos_dopp_shift}
    f_D = f_0 \frac{1}{c} (\Vec{v_T} - \Vec{v_R}) \frac{\Vec{r_T} - \Vec{r_R}}{||\Vec{r_T} - \Vec{r_R}||}
\end{equation}

The assumption of stationary receiver requires the positions to be in ECEF frame, as otherwise the receiver rotates with the Earth.



\section{The Doppler method}
\label{s_pos_doppler_method}
The Doppler method is a method of radio-frequency navigation which calculates the user position from measurements of the Doppler shift of the received signal and from the knowledge of the transmitter position.

When a satellite passes over the surface of the Earth with a tangential velocity $v$, transmitting a signal on a frequency $f_0$ (see \autoref{f_pos_doppler_method_simple}), a user on the surface measures a Doppler-shifted frequency $f_R$ (see \autoref{e_pos_dopp_fr}).

\begin{figure}
    \centering
    \includegraphics[width=0.75\linewidth]{img/pos_doppler_method_simple}
    \caption{The Doppler method principle}
    \label{f_pos_doppler_method_simple}
\end{figure}

The relative velocity of the satellite and the user, $v_T$, is the component of the satellite velocity vector, which points in the direction of user. It is offset from the satellite velocity by the angle $\alpha$. It can be expressed as

\begin{equation*}
    v_T = v \ \cos{\alpha}
\end{equation*}
Thus, the Doppler shift can be expressed as
\begin{align}
    \label{e_pos_fd_alpha}
    f_D &= f_R \frac{v}{c} \cos{\alpha} \\
    \label{e_pos_cos_alpha}
    \text{thus } \cos{\alpha} &= \frac{f_D}{f_R \ v}c
\end{align}

The user is located somewhere on an infinite cone (the \textit{isodoppler cone}) with the centre at the position of the satellite at the time of transmission, the axis in the direction of $v$ and an opening angle $2\alpha$.
% "cone" - ne rotační hyperboloid? Možná v angličtině je to jinak než v ČJ. "Cone" jste našel v literatuře? Pokud ano, asi OK. Měl jsem vždy dojem, že jsme tomu v češtině říkali rotační hyperboloid, ne kužel. Ale možná je to opravdu kužel, ne hyperboloid, a říkal jsem nesmysly.
%% on to není hyperboloid, je to skutečně kužel. V literatuře to je, akorát ne u rodilých mluvčích

If the measurement is carried multiple times, a \textit{Doppler curve} can be obtained - a curve of Doppler shifts in time ($t$, $f_D$). The user is located at the intersections of the respective isodoppler cones. If the user is assumed to be on the surface, an intersection with a sphere representing the surface can be considered.

Furthermore, it is apparent from \autoref{e_pos_cos_alpha}, that if $f_D = 0$, $cos \alpha = 0$ and therefore $\alpha = \ang{90}$ - the isodoppler curve at that point becomes a plane perpendicular to the satellite orbital velocity. When projected on the surface of the Earth, the plane appears as a line perpendicular to the satellite track.

It is important to note that for a pass of a single satellite, the Doppler method always yields at least two results (one actual and one shadow) symmetrical along the satellite ground track (see \autoref{f_pos_doppler_symmetry_problem} or \autoref{f_des_symmetry} for the symmetrical solution being found in practice). Additional satellite measurements or other means are necessary to resolve the symmetry problem and determine which position is the correct one.

\begin{figure}
    \centering
    \includegraphics[width=0.75\linewidth]{img/pos_doppler_symmetry_problem}
    \caption{Illustration of the Doppler method symmetry problem\cite{sop09}}
    \label{f_pos_doppler_symmetry_problem}
\end{figure}



\section{Implementations the Doppler method}
A review of existing work reveals two general ways of implementing the Doppler method. They are based on variations of \autoref{e_pos_dopp_shift}.

\subsection{The Doppler curve fitting method}
\label{s_pos_curve_fit_method}
The Doppler curve fitting method is based on measuring the frequency emitted by a satellite, thus obtaining a Doppler shift curve. This measured curve is then compared using the least-squares fit to the trial Doppler curves generated based on trial user position. The trial position is corrected until the best fit is found. The trial Doppler curve is calculated based on \autoref{e_pos_dopp_shift}. This method had been used by the Transit navigation system\cite{sat16}.


\subsection{The Observation Model method}
The second method is the most common in the reviewed research. It is based on creating an observation model and injecting Doppler measurements into an Extended Kalman Filter (EKF) to estimate the user location\cite{sop02, sop03, sop13}.
% Super - díky za uvedení i této metody
%% :)

The range rate measurement of $i$\textsuperscript{th} satellite was usually modelled in the research as
\begin{equation}
   \dot\rho_i = \frac{c f_{D_i}}{f_R} = (V_i - V) \frac{r_i - r_0}{||r_i - r_0||_2} + c \dot\delta t - c \dot \delta t^i + \sigma
\end{equation}
where $c \dot\delta t$, $c \dot \delta t^i$ is the clock drift in \unit{\m\per\s} of the receiver and satellite, respectively, and $\sigma$ is Gaussian white noise\cite{sop02}. 


\section{Maximum Doppler shift}
An estimate of the maximum of Doppler shift which can occur in a captured signal frequency is a useful parameter for signal acquisition and for transmission channel identification, where it can serve as a limit.

The largest shift occurs when $\cos{\alpha} = 1$ in \autoref{e_pos_fd_alpha}, that is, when the satellite radius vector is perpendicular to that of the receiver. However, in this situation, the satellite in LEO is invisible to the receiver. More practically, the largest Doppler shift occurs when the satellite is first (and last) visible to the receiver. For a ground-based receiver with no obstacle in sight, this happens when the satellite intersects a plane tangential to the Earth surface at the receiver position.

In this special case, the angle $\alpha$ is equal to the angle between the satellite and user radius vectors. Thus,
\begin{equation*}
    \cos{\alpha} = \frac{R_E}{R_E + h}
\end{equation*}
where $R_E$ is the radius of the Earth and $h$ is the orbital altitude of the satellite.

Orbital velocity for a circular orbit is approximately
\begin{equation*}
    v \approx \sqrt{\frac{\mu}{R_E + h}}
\end{equation*}
where $\mu$ is the standard gravitational parameter. Thus, substituting the equation above into \autoref{e_pos_fd_alpha}, the maximum Doppler shift depends only on the orbital altitude and transmission frequency:
\begin{equation}
    \label{e_pos_fd_max}
    f_{D max} \approx f_R \frac{1}{c} \sqrt{\frac{\mu}{R_E + h}} \frac{R_E}{R_E + h}
\end{equation}

The calculated maximum Doppler shift for the satellites considered in the LEO system survey in \autoref{s_sat} is in \autoref{t_pos_max_fd}.
% OK. Ty vzorce vypadají opět hezky. Odvozeno, nebo někde z literatury? Ptám se jen kvůli možnosti kontroly. Stejně jako tabulka 2.1.
%% odvozeno a spočítáno mnou, přidal jsem popisky

\begin{table}
    \centering
    \begin{tabular}{llll}
    System     & $h$ (km) &  $f_R$ (MHz) & $f_{Dmax}$ (kHz) \\ \hline
    Iridium    &  781  &  1626 & \num{36.1} \\
    Orbcomm    &  715  &  138  & \num{3.1} \\
    Globalstar &  1414 &  2500 & \num{48.9}
    \end{tabular}
    \caption{Maximum Doppler shift for select LEO satellites (calculated)}
    \label{t_pos_max_fd}
\end{table}



\section{Capturing a satellite signal}
\label{s_pos_tracking_satellite}
One of the basic principles of capturing a satellite signal burst, extensively used in the reviewed literature, is based on thresholding the signal power level.

First, the raw captured signal is amplified by a low noise amplifier (LNA) and is squared once or twice (2\textsuperscript{nd} or 4\textsuperscript{th} power) to increase the prominence of dominant peaks. Then, the Fast Fourier transform (FFT) or a Power Spectral Density (PSD) function is applied with a windowing function (e.g. Hamming or Blackman-Harris). The resultant spectrum is then compared to a threshold representing the noise level. Peaks whose power level is stronger than the threshold and whose frequency falls within the plausible range for the given signal type are considered to be indicative of a present signal. The peak rise and fall times are noted (usually in terms of sample number rather than seconds). 

The samples within those times enter a Phase Lock Loop (PLL) to refine the frequency estimate. The PLL is composed of a Numerically Controlled Oscillator
(NCO), integrator functions, phase detector and a loop filter. A typical setup is shown in the block diagram in \autoref{f_pos_tracking_block}\cite{sop03, sop04, sop05}.

\tikzstyle{block} = [rectangle, rounded corners, minimum height=1cm, text centered, text width=1cm, draw=black]
\tikzstyle{sum} = [draw, circle, minimum size=0.6cm, fill=black!15]
\tikzstyle{empty} = [draw=white, circle, minimum size=0.6cm, ]
\tikzstyle{arrow} = [thick,->,>=stealth]

\begin{figure}
\centering
\begin{tikzpicture}[node distance=2cm]

% Sum shape
\node (ant) [empty]{};
\draw [very thick] (ant.north east) -- (ant.center)
      (ant.north west) -- (ant.center)
      (ant.north east) -- (ant.north west)
      (ant.north) -- (ant.south);
\node (antb) [below of=ant]{};
\node (lna) [block, right of=antb] {LNA};
\node (pwr) [block, right of=lna] {$x^2$/$x^4$};
\node (fft) [block, right of=pwr] {FFT/ PLL};
\node (thr) [block, right of=fft] {$>$ dB};
\node (sm1) [sum,   right of=thr]{};
\draw (sm1.north east) -- (sm1.south west)
      (sm1.north west) -- (sm1.south east);
\node (pll) [block, right of=sm1] {PLL};
\node (pllr)  [right of=sm1, xshift=1.5cm]{};

\draw (ant) -- (antb.center);
\draw [arrow] (antb.center) -- (lna);
\draw [arrow] (lna) -- (pwr) node[midway](bypass){};
\draw [arrow] (pwr) -- (fft);
\draw [arrow] (fft) -- (thr);
\draw [arrow] (thr) -- (sm1) node[midway,above]{$t_0, t_1$};
\draw [arrow] (sm1) -- (pll);
\node (sm1b) [below of=sm1]{};
\node (lnab) [below of=bypass]{};
\draw (bypass.center) -- (lnab.center);
\draw (lnab.center) -- (sm1b.center);
\draw [arrow] (sm1b.center) -- (sm1);
\draw [arrow] (pll) -- (pllr) node[midway,above]{$f_D$};
\end{tikzpicture}
\caption{General process of capturing a satellite signal}
\label{f_pos_tracking_block}
\end{figure}
% Zde, jako radioelektronik bych měl asi spoustu otázek. Ale lidi %mimo to nenapadne. Stejný obrázek 2.3 byl i v literatuře? Otázka %by byla, co tam dělá to násobení, a jestli PLL je skutečně %zavěšeno na tom signálu po kvadrátu, a zda je to skutečně PLL, %nebo i FLL. Mluvíte o těchto věcech někde dál? Pokud ne, mělo by %být zmíněno že tohle řeší už nějaký hotový SW. Ale citace %16,17,18 tam jsou, ok.
%% tohle je obecné, objevuje se to v několika zdrojích - pro můj kód je to v Designu


\section{Calculating satellite position}
\label{s_pos_tle_sgp4} 
The Doppler method requires the knowledge of the position of the transmitting satellite, which is then used as a reference. Unless the satellite itself transmits its position (which is the case for Iridium Ring Alert messages, albeit in a low resolution), the 
% to malé rozlišení - bude někde ještě zmíněn ten rozdíl co mezi ním a TLE od Noradu je, a jak jste na něj přišel?
%% je to ukázáno v Design
position needs to be calculated on the user side. To do this, the satellite orbit needs to be known and a propagation model used to calculate the satellite position at a given time.

A widely used approach utilises Two Line Element (TLE) sets, which contain orbital parameters for satellites, and the Simplified General Perturbations-4 (SGP4) model for propagation. TLEs are regularly updated by the North American Aerospace Defense Command (NORAD) and are publicly available via e.g. the CelesTrak project website\cite{des11}. An analysed example of TLE is shown in \autoref{f_pos_tle}. A TLE is valid for a given time, which is called an \textit{epoch}.

\begin{figure}
    \centering
    \includegraphics[width=1\linewidth]{img/pos_tle.PNG}
    \caption{Structure of a TLE set\cite{pos06}}
    \label{f_pos_tle}
\end{figure}

The SGP4 model is an orbit propagation model developed by the U.S. Air Force in the 1960s and further refined e.g. by Vallado et al.\cite{des06}. It is compatible with TLEs and commonly used together. The perturbation calculations are simplified - the Earth’s gravitational model is truncated, the atmospheric model is a static density field with exponential decay, and third-body influences are modelled only partially\citep{pos01}{698}. The SGP4 model works in the TEME coordinate frame (see \autoref{s_pos_frames_of_ref}).

The accuracy of the model was evaluated in \cite{pos07}. It was concluded that for a LEO satellite, the position prediction errors of the SGP4 algorithm when the TLEs are one day old are on the order of \qty{e2}{m}, with maximum errors between \qtyrange{3}{6}{km}, depending on the intensity of the Solar flux.



\section{Frames of reference}
\label{s_pos_frames_of_ref}
Any position needs to be expressed in a frame of reference. For terrestrial calculations, it is advantageous to use an Earth-centred frame, which has its origin at some centre (e.g. gravitational, physical) of the Earth. There are two general types of Earth-centred frames: Earth-Centred Inertial (ECI), which are fixed relative to some position external to the Earth (usually the Earth's position at equinox) and thus do not rotate with the Earth (they are inertial), and Earth-Centred Earth-Fixed (ECEF), which are fixed to a point on Earth and thus do rotate with it (they are non-inertial).

For this work, three frames of reference are important: TEME (ECI), ITRS (ECEF) and WGS84 geodetic datum (ECEF).

The process of transforming between those systems is described e.g. in \cite{pos01}, and the transformations used in this work are all based on the equations and algorithms provided there: from TEME to ITRS \citep{pos01}{231-233}, from ITRS to geodetic \citep{pos01}{174-179} and back \citep{pos01}{146}.

\subsection{True Equator Mean Equinox}
The True Equator Mean Equinox (TEME) reference frame is a Cartesian ECI frame. The z-axis points along the Celestial Ephemeris Pole\footnote{The Earth rotation axis} (CEP), while x-axis is located on the True Equator plane (a plane perpendicular to the CEP) and points in the direction of the Vernal Equinox without accounting for celestial nutation (the \textit{mean} or \textit{uniform} equinox)\citep{pos01}{231-233}. The TEME frame is used by the SGP4 algorithm (see \autoref{s_pos_tle_sgp4}).

\subsection{International Terrestrial Reference System}
The International Terrestrial Reference System (ITRS) is a standard Cartesian ECEF frame. The origin is at the centre of mass of the Earth and the axes are fixed to defining coordinates on Earth. As a result of tectonic movement, the system is periodically recalculated\citep{pos01}{152}. All calculations within this work are done in ITRS.

\subsection{World Geodetic System Geodetic Datum}
The World Geodetic System-84 (WGS84), where 84 refers to the most recent version from 1984, is an ECEF frame maintained by the U.S. National Geospatial-Intelligence Agency. The WGS84 and ITRS agree on a level of centimetres\citep{pos01}{203}. The geodetic datum  (hereafter referred to as only the \textit{geodetic frame}) is a latitude-longitude-altitude ($\phi, \lambda, h$ respectively) frame defined within WGS84. In this work, the estimated position is in the geodetic frame for clarity, however, for navigation calculations themselves the position is always converted to ITRS.

% Opět, výborně.
%% Díky :)
\chapter{A survey of current LEO satellite systems}

\section{Iridium NEXT}
Iridium satellite constellation, owned and operated by Iridium Communications Inc., provides global communications coverage for satellite phones, pagers and other devices, as well as an interface to terrestrial communications networks via several Gateways.

The system became operational in 1998 and saw a total of 96 satellites launched, before the constellation was completely overhauled between 2017 and 2019, replacing all of the original satellites with Iridium Next satellites. Currently, no original satellite remains operational, however, 25 satellites remain defunct on orbit\cite{sat06}.

The constellation consists of 66 satellites, 11 each in six orbital planes, spaced \qty{30}{\deg} apart. The altitude of the satellites is approx. \qty{781}{\km}, the inclination is \qty{86.4}{\deg} and the orbital period is approx. \qty{100}{\min}. For a ground user, one satellite is visible for about \qty{7}{\min}. The constellation covers the entire global surface\cite{sat01}.

Communication is carried over Ka-band for satellite-satellite links and satellite-gateway links, and over L-band (\qtyrange{1616}{1626.5}{\mega\hertz}) for satellite-user links. Each satellite provides 48 individual spot beams, sharing 240 traffic channels with a frequency re-use pattern\cite{sat07}.

The Iridium Next onboard clocks have a stability better than \num{10e-9} and a drift less than \qty{3}{\hertz\per\s}, which is sufficient for time transfers with accuracy in the order of \qty{e0}{\micro\s}\cite{sop11}.

\subsection{Iridium signals}
Due to the scope of this work, only the satellite-user (L-band) signals are examined, as the Ka-band signals are either not globally transmitted or not aimed towards the Earth.

Iridium uses a combination of Space Division Multiple Access (SDMA),  Frequency Division Multiple Access (FDMA), Time Division Multiple Access (TDMA) and Time Division Duplex (TDD) multiple access schemes\cite{sop11}. For every spot beam, channels\footnote{A channel is a specific FDMA frequency and TDMA timeslot\cite{sat07}} are implemented using TDMA architecture based on TDD using a time frame\cite{sop12}. The structure of a TDMA frame is illustrated in \ref{sat_iridium_freq_and_frame_structure}.

In the \qtyrange{1626}{1626.5}{\MHz} band Iridium uses, there are 252 carriers with carrier spacing of \qty{41.667}{\kHz}, grouped into 31 sub-bands of 8 and one of 4. There are 5 simplex carriers in the band with a spacing of \qty{35}{\kHz}, one of which is the Ring Alert, and the remaining four are messaging carriers for paging and acquisition. The polarisation of both the uplink and downlink band is right-hand circular\cite{sat04}.

\begin{figure}
    \centering
    \includegraphics[width=0.75\linewidth]{img/sat_iridium_freq_and_frame_structure.png}
    \caption{Iridium frequency allocation and frame structure\cite{sop12}}
    \label{sat_iridium_freq_and_frame_structure}
\end{figure}

\subsubsection{Paging}
Paging signals enable users to receive ringing and paging messages during heavier atmospheric fading conditions and in buildings where attenuation is greater. There are five paging channels, one for alert (Ring Alert, detailed below) and five for transmitting paging messages to the receive-only terminals.  The transmission duration does not exceed \qty{20.32}{\ms}\cite{sat09}.

\subsubsection{Ring Alert}
The Iridium Ring Alert signal is an unencrypted downlink-only simplex channel with a carrier frequency of \qty{1626.2708}{\MHz}\cite{sat04}. Each of the 48 satellite beams transmits a Ring Alert message every \qty{4.32}{\s}, for a transmission period of \qty{90}{\ms} of one satellite\cite{sat07}, which can be used to keep track of a specific satellite\cite{sop11}.

The transmit power for the Ring Alert channel is higher than that of voice/data channel, so as to enable the mobile earth terminals to receive ring alerts even when their antennas are stowed\cite{sat09}.

The Ring Alert Message contains the following information\cite{sat08}:
\begin{itemize}
    \item Satellite ID - a numeric identifier (\numrange{2}{115}) of the transmitting satellite, which notably does \emph{not} correlate with the satellite NORAD identifier,
    \item Beam ID - a numeric identifier (\numrange{0}{47}) of the transmitting beam,
    \item Latitude - the current \emph{ground} latitude of the satellite in degrees, with two decimal places of precision, calculated by the satellite,
    \item Longitude - same format as Latitude and
    \item Satellite altitude - altitude above the surface in \unit{\km}, which was shown in \cite{sat08} to be unreliable.
\end{itemize}

Importantly, the inclusion of satellite ID in the Ring Alert message, combined with the fixed and known frequency of the Ring Alert channel and the absence of encryption, enables any receiver to quickly determine which satellite is transmitting. The Ring Alert signals are thus very useful for SoP navigation using Iridium.

\section{Orbcomm}
much info in \cite{sop08}

\section{Comparison of system parameters}
\chapter{Survey of current SoP positioning systems}

\section{GNSS}

\section{Doppler-shift positioning system}
\subsection{Transit}
The Navy Navigation Satellite System, commonly referred to as Transit or NAVSAT, was the first operational satellite navigation system. The system was used to provide accurate position updates for the INS equipment aboard Polaris nuclear ballistic missile submarines. Later, civilian use cases such as fishing boat navigation emerged. The system was operational from 1964. The system used the Doppler method for positioning, the details of which are described in Section \ref{s_transit_nav_method}. The transmission frequencies were \qtylist{150;400}{MHz} (two to allow for compensation of ionospheric refraction) and the system also provided a data message used for clock synchronisation. The performance of the system exceeded that of all of the early large radio systems - the RMSE of a position fix of a static naval user was \qtyrange{40}{100}{m} and the time to fix was \qtyrange{10}{16}{min}, but could be as low as \qty{2}{min} with lower accuracy\cite{sat16}.

\subsection{Argos}
Argos is a global satellite-based location and data collection system dedicated to studying and protecting Earth environment, operated through a multinational research agency collaboration. It allows any mobile object, usually a tracked animal, equipped with a compatible transmitter terminal to be located across the world\cite{sat17}. Unlike all other systems discussed, the actual position determination is done at the level of the satellite, which receives a signal (which may also contain sensor data) from a terminal and calculates its position.

Argos terminal messages are transmitted at a frequency of \qty{401650(30)}{kHz}. Calculation of terminal position is done through the Doppler shift method, using a classical nonlinear least squares estimation or a multiple-model Kalman filter. The positioning accuracy typically ranges from \qty{e1}{km} to less than \qty{250}{m}, depending on the number of terminal messages received during a satellite pass\cite{sop09}. Algorithms based on animal movement models are used to resolve the symmetrical problem of Doppler positioning\cite{sat17}.


\section{Experimental SoP positioning systems}
This section contains a survey of select experimental SoP positioning systems, each with a brief description of the implementation and experimental results. All of these system use the Doppler-shift method.

%todo zkratky
In \cite{sop07}, it is demonstrated using analysis and simulation that it is possible to obtain quick\footnote{Definition of "quick" is not provided, but it is stated that it is significantly less than \qty{16}{\s}} positioning using Doppler frequency measurements of signals from one or two satellites of the Globalstar constellation (forward down-link carrier frequency of \qty{2500}{\mega\hertz}). The error in the horizontal plane of such positioning is expected to be less than \qty{9}{\km} in \qty{90}{\percent} of cases using one satellite. Using two satellites, the error is expected to be \qty{1.4}{\km} in \qty{90}{\percent} of cases. This is without any advanced signal processing or data fusion.

In \cite{sop01}, a method of positioning is devised for use as a secondary aircraft navigation system in GPS-denied areas using Iridium Next satellites and SDR. CEP of \qtyrange{0.2}{2}{\km} in dynamic mode is demonstrated without using significant computational resources. The system is based on receiving STL, Ring Alert and MSG Iridium Next signals on \qty{1626.104}{\mega\hertz}, \qty{1626.27}{\mega\hertz} and \qty{1626.44}{\mega\hertz}, respectively, and then fusing the Doppler shift data with IMU data using Kalman filtering.

In \cite{sop04}, a Starlink-based system is designed, based on Doppler measurements of a tone band centered on \qty{11 325}{\mega\hertz}. The system tracked the tone signal of up to six satellites by FFL-assisted-PPL. Extended Kalman filtering was used. The system accuracy was \qty{375}{\metre}. The team determined that the most significant error sources were the receiver (user) clock drift, the user time error and the satellite position error. Notably, the satellite clock drift was not determined to be significant.

In \cite{sop05}, a system based on the fusion (using EKF) of Iridium Doppler shift measurements, INS and magnetometer data is proposed and simulated for a highly dynamic application. The simulated Iridium signal was centred on \qty{1626.25}{\mega\hertz}. The arrival position error\footnote{That is, the error in position after travelling along a predefined trajectory in the simulation} in \qty{95}{\percent} of cases was approximately \qty{300}{\metre} for the best case, and \qty{4}{\km} for the intermediate case. The main difference between the two cases was the accuracy of orbital data.

In \cite{sop08}, a framework is presented which estimates the LEO satellite position states along with the states of a navigating vehicle using Orbcomm satellite signals. The system fuses INS with Doppler measurements of a signal in VHF downlink band of \qtyrange{137}{138}{\mega\hertz} using EKF. The system was tested using an UAV, which had GPS signal available for \qty{90}{\s} and then turned off for \qty{30}{\s}. The final position error was \qty{8.8}{\m}, compared with INS-only error of \qty{31.7}{\m}. Static positioning without the use of GNSS was not performed.

In \cite{sop10}, a framework employing EKF to estimate user position from Doppler measurement of multiple generic LEO satellites was proposed. Simulation showed a RMSE in position from approx. \qty{170}{\m} for 5 satellites over one minute to \qty{11.5}{\m} for 25 satellites over four minutes. Experimental run with 2 Orbcomm satellites (broadcasting at \qty{137.3125}{\mega\hertz} and \qty{137.25}{\mega\hertz} respectively) over one minute showed a RMSE of \qty{360}{\m}.

In \cite{sop11}, and Iridium-based system is proposed and tested. The system uses the Ring Alert and Primer message signals to capture Doppler shift data, which are then processed using Kalman filtering and the least squares method. The system accuracy can be as much as \qty{22}{\m} after \qty{24}{\hour} of measurement under an open sky. For \qty{30}{\min} under open sky, the accuracy was shown to be \qty{46}{\m}. Furthermore, the system accuracy in GPS-hostile conditions (in a dense forest) was shown to be \qty{108}{\m}\footnote{The experiment time is not stated, but can be inferred to be about \qty{30}{\min}}, whereas GPS did not provide a solution.

In \cite{sop03}, two LEO constellations (Iridium Next and Orbcomm) are used to determine position. Three dimensional RMSE achieved over \qty{30}{\s} using a single constellation (Orbcomm) is approx. \qty{0.76}{\km}, whereas using both constellations the RMSE is \qtyrange{0.22}{0.18}{\km} depending on the received signal type. Accuracy when measuring messaging bursts was better than for Ring Alert bursts. EKF was utilised in the signal processing. In the receiver architecture, each constellation has a dedicated acquisition chain, including separate antennas, amplifiers, filters and signal processing. 

The research outlined above is summarised in table \ref{t_sop_survey_summary}. Most research focused on using Iridium Next satellites as SoP source, followed by Orbcomm. Accuracies of position determination were mostly in the order of \qty{e2}{\m} and improved significantly both with measurement time and with the number of measured satellites. In one case the performance was shown to exceed the one of existing GNSS systems. Furthermore, the surveyed research demonstrated the possibility of using SoP positioning both in static and dynamic applications. Notably, all of the surveyed experiments used Kalman filtering for fusion of multiple navigation system data or error estimation, albeit the actual implementation varied greatly.

\begin{table}
\caption{Summary of selected existing SoP positioning systems (* denotes a simulated result)}
\label{t_sop_survey_summary}
\hspace*{-2cm}
\centering
\begin{tabular}{p{0.2\linewidth}p{0.15\linewidth}lllll}
Constellation          & Signal                             & Mode       & Accuracy (2D)                  & Exp. time               & Source  \\ \hline
Globalstar             & \qty{2500}{\mega\hertz}            & static     & \qtyrange{1.4}{9}{\km}*        & few s                   & \cite{sop07}     \\
Iridium Next           & STL, Ring Alert, MSG               & dynamic    & \qtyrange{0.2}{2}{\km}         & N/A                     & \cite{sop01}     \\
Starlink               & \qty{11 325}{\mega\hertz}          & static     & \qty{375}{\metre}              & \qty{330}{\s}           & \cite{sop04}     \\
Iridium Next           & \qty{1626.25}{\mega\hertz}         & dynamic    & \qtyrange{0.3}{4}{\metre}      & N/A                     & \cite{sop05}     \\  % fixme Maybe remove?
Generic, Orbcomm       & \qty{137}{\mega\hertz}             & static     & \num{11.5}* to \qty{360}{\m}   & \qtyrange{1}{4}{\min}   & \cite{sop10}     \\
Iridium Next           & Ring Alert, Primer                 & static     & \qtyrange{46}{108}{\m}         & \qty{30}{\min}          & \cite{sop11}     \\
Iridium Next, Orbcomm  & MSG, Ring Alert                    & static     & \qtyrange{0.18}{0.76}{\km}     & \qty{30}{\s}            & \cite{sop03}    
\end{tabular}
\end{table}


\section{Comparison to GNSS}
% todo

\chapter{Design}
% todo intro

The navigation system works in several steps, each of which is discussed in detail below:
\begin{enumerate}
    \item Signals from satellites are captured,
    \item data frames are extracted from the signal and decoded,
    \item decoded frames are processed - transmitting satellite is identified, the satellite position at the transmission time is calculated and frequency curves for individual satellites are found,
    \item initial location is estimated, and user position is calculated.  
\end{enumerate}


\section{Capturing signals}
The first step in determining the user position is to capture signals transmitted by the satellites used for navigation. To achieve that, an antenna and a radio are needed.

\subsection{Antenna setup}
As discussed in section \ref{s_sat_iridium_signals}, Iridium NEXT transmits in the \qtyrange{1626}{1626.5}{\MHz} frequency range, which lies within the L-band. The signal strenght on Earth is expected to be in the \qtyrange{-90}{-110}{dBm} range\cite{sop01}, which is in the order of \unit{pW} in absolute signal power.

Thanks to the relative proximity (in frequency terms) of the Iridium signals to those of GPS, a GPS antenna can be used. However, some active antennas, such as the NovAtel GPS-702\cite{des03}, are severely attenuated in the Iridium signal band, and thus cannot be used.

The antenna used in this work is NovAtel GPS-704-X passive GNSS antenna. 
The 3 dB pass band of this antenna is \qtyrange{1.15}{1.65}{GHz} with minimum gain at zenith of \qty{+6.0}{dBic} in the L1 band. Additionally, the antenna features NovAtel’s patented Pinwheel technology for multipath rejection and phase centre stability\cite{des04}. The antenna was chosen due to its suitable characteristics as well as its availability.

To acquire Iridium signals of sufficient strength, a low noise amplifier was found to be necessary. The amplifier used in this work is the Mini-Circuits ZX60-3018G-S coaxial amplifier with frequency range of \qtyrange{0.2}{3}{GHz}, typical maximum output power of \qty{+12.8}{dbm} and gain of \qty{20.60}{dB} for \qty{1671}{MHz}\footnote{Closest value in the datasheet.}\cite{des06}.

\subsection{Radio}
The radio used in the navigation system is the ADALM-PLUTO (in short PlutoSDR) software defined radio development kit. It is a portable self-contained RF-learning module, equipped the AD9363 RF Agile Transceiver. PlutoSDR has one TX and one RX channel with a RF range of \qtyrange{0.325}{3.8}{GHz}. The maximum bandwidth is \qty{20}{Mhz}. The SDR is powered and communicates through USB2.0 with drivers available in Windows and application programming interface (API) in, among others, C, C++ and Python. The APIs are implemented in several development platforms, notably Matlab and GNURadio (via SoapySDR). The reference clock of the SDR has a nominal frequency of \qty{40}{Mhz} and an accuracy of \num{\pm25e-6}\cite{des05}. The block diagram of the SDR is in figure \ref{f_des_pluto_block}.

\begin{figure}
    \centering
    \includegraphics[width=\linewidth]{img/des_pluto_block.png}
    \caption{PlutoSDR block diagram\cite{des05}}
    \label{f_des_pluto_block}
\end{figure}

\subsubsection{Frequency offset and drift}

The synthesizer blocks that generate all data clocks, sample clocks, and local oscillators inside the PlutoSDR transceiver are supplied by a reference clock, which is external to the transceiver, but it is integrated in PlutoSDR and connected to pin \texttt{XTALN} of the transceiver (see fig. \ref{f_des_transciever_diag})\cite{des07}. Due to frequency conversion, any error in the frequency of the reference clock will be multiplied in actual measurements. Given the reference clock nominal frequency of \qty{40}{Mhz} is approximately 40x times lower than the measured band (\qty{\approx1600}{Mhz}), \qty{1}{Hz} offset in reference clock frequency will appear as \qty{-40}{Hz} shift in measured frequency.

\begin{figure}
    \centering
    \includegraphics[width=0.75\linewidth]{img/des_transciever_diag.PNG}
    \caption{AD9363 functional diagram\cite{des07}}
    \label{f_des_transciever_diag}
\end{figure}

Measurements of satellite signals showed that the reference clock of the PlutoSDR suffers from significant frequency offset, which varies with time. As a result, the measured signal frequency differs from actual received frequency by up to \qty{20}{kHz}. A significant cause of the change in the reference clock frequency are variations in temperature, both as a result of internal heating of the SDR components and outer environment changes, which cannot be avoided when measuring outdoors or over a long period of time.

An experiment was carried out, in which the PlutoSDR reference clock frequency was compared to a precise frequency generator (an atomic clock at the FEE CUT). After SDR startup, the reference clock frequency was \qty{39999729}{Hz} (nominal frequency  $f_N$ is \qty{40000000}{Hz} - a difference of \qty{-271}{Hz} or $\num{-6.8e-6} \times f_N$). After 20 minutes of operation, the reference clock frequency was  \qty{39999571}{Hz} (\qty{-429}{Hz} or $\num{-10.7e-6} \times f_N$). After this time, the frequency varied by \qty{\pm 50}{Hz}, but mostly stabilised. Any change in external environment, such as opening the laboratory window, caused significant variation in reference clock frequency.

As a result of this experiment, the PlutoSDR was covered in polystyrene case when measurements were taken, to somewhat stabilise its temperature. Additionally, the SDR was calibrated to account for some of the discrepancy in reference clock frequency, by setting the device to assume the reference clock frequency is \qty{39999571}{Hz}.

This experiment, as well as the satellite signal measurements shows the navigation system needs to account for frequency offset (henceforth referred to as just \textit{offset}) and the change of this offset (henceforth referred to as \textit{drift}) in its calculations. Without precise external frequency source, such as atomic clock (which is unfeasible) or a GPS-synchronised oscillator (which introduces a reliance on GNSS), precise measurements of absolute frequency are not possible. Using a better SDR with more precise or better temperature-compensated oscillator might help, but any system needs to handle unreliable frequency measurements by itself.

\subsection{Measurement setup}
For successful signal capture, the antenna needs to have a wide and mostly unobstructed field of view. Furthermore, the radio should be appropriately protected from rapid changes in environment temperature (such as wind), and adequate power supply must be provided to the radio.

An example of the measurement setup used in this work is photographed in figure \ref{f_des_meas_setup}. It consist of the GPS-704-X antenna with the amplifier (1), an antenna stand (an orange planting box with a bit of cardboard on top) (2), the PlutoSDR in polystyrene case (weighed down by nice rocks the author brought from Greece) (3), a \qty{12}{VDC} laboratory power supply (which is older than the author) (4) powering the amplifier, a Lenovo T440p laptop (5) running the measurement software and a power supply (a slightly rotten power outlet whose circuit breaker the author found after 30 minutes of searching in the \textit{basement}) (6). The measurement setup changed very little during the course of the work.

The view of the antenna is unobstructed above approximately \ang{15} of elevation, with the exception of north direction, which is slightly obstructed by a tree. However, as is apparent in the data, the field of view is overall satisfactory.

\begin{figure}
    \centering
    \includegraphics[width=1\linewidth]{img/des_meas_setup}
    \caption{Measurement setup}
    \label{f_des_meas_setup}
\end{figure}


\section{Frame extraction and decoding}
One of the methods of acquiring and tracking a satellite signal has been described in section \ref{s_pos_tracking_satellite}. However, that refers only to acquiring the signal itself and measuring its frequency. In this navigation system, it is highly advantageous to also demodulate and decode the data within the signal, for instance to aid in satellite identification.

Extracting, demodulating and decoding frames from satellite signal is a specialised task, which is beyond the scope of this work. Therefore, two already existing programs were used - \texttt{gr-iridium}\cite{des09} and \texttt{iridium-toolkit}\cite{des10}.

The \texttt{gr-iridium} package is a module for GNURadio, which handles capturing and demodulating frames sent by Iridium satellites. It works with generic SDRs and outputs decoded Iridium frames as text output in console. Provided it is supplied with proper configuration, the package works from command line and is Windows-compatible, with compiled binaries available in Conda package manager. It requires GNURadio 3.10.
 
The \texttt{iridium-toolkit} is a Python application capable of decoding demodulated Iridium frames captured by \texttt{gr-iridium}. It can work from command line, is Windows-compatible and does not require compilation nor it relies on GNURadio. It can decode the data in several types of Iridium frames.

Both of these packages have already been used in research, such as in \cite{sat08} to capture Iridium Ring Alert messages over a long time.

\paragraph{A note on nomenclature:} In this context, \textit{extracting} frames means acquiring information about a transmitted frame that do not go beyond its RF properties, such as center frequency, signal strength or time of arrival. \textit{Demodulating} means converting captured samples into data bits, while \textit{decoding} means converting the bits into meaningful (and human-readable data).

\subsubsection{Process used by \texttt{gr-iridium} to capture and decode Iridium frames}
Below is a description of the process the \texttt{gr-iridium} uses to capture and demodulate Iridium frames. Since the documentation of the package is sparse, the information below comes was inferred from code. References to specific sections of code are in footnotes where appropriate\footnote{The source code is available at \url{https://github.com/muccc/gr-iridium/blob/v1.0.0/}}.

The core of the package is a GNURadio flowgraph (see fig. \ref{f_des_gr_iridium_blocks}\footnote{The flowgraph was created by the author, it is not actually present in graphical form in the package}). Within this GNURadio environment, IQ samples are passed along the solid arrows into the blue nodes of the function blocks and Protocol Data Unit (PDU) messages are passed along the dotted arrows into the grey nodes. The process of extracting and demodulating frames is sequential - it starts in the \texttt{Soapy Custom Source} and ends in the \texttt{iridium\_frame\_printer}.

% command to ease citing of code in gr-iridium
%\newcommand{\grline}[2]{\footnote{gr-iridium v1.0.0: lib/#1\_impl.cc: line #2, available at \url{https://github.com/muccc/gr-iridium/blob/v1.0.0/lib/#1\_impl.cc\#L#2}}}
\newcommand{\grline}[2]{\footnote{gr-iridium/lib/#1\_impl.cc, line #2}}
\newcommand{\grlinetagger}[1]{\grline{iridium\_burst\_tagger}{#1}}
\newcommand{\grlinedownmix}[1]{\grline{burst\_downmix}{#1}}
\newcommand{\grlinedemod}[1]{\grline{iridium\_qpsk\_demod}{#1}}


\begin{figure}
    \centering
    \includegraphics[width=1\linewidth]{img/des_gr_iridium_blocks}
    \caption{\texttt{gr-iridium} flowgraph}
    \label{f_des_gr_iridium_blocks}
\end{figure}

The \textbf{\texttt{Soapy Custom Source}} block handles communication with the PlutoSDR. It sends out raw IQ samples for further processing.

The \textbf{\texttt{iridium\_burst\_tagger}} block identifies Iridium bursts in the captured data and "tags" them by creating metadata with references to burst start time (in terms of samples), magnitude, internal ID, centre frequency, rough carrier frequency offset (CFO) etc. First, the block squares the received signals twice\grlinetagger{527}. Then it performs FFT on the received samples with as fixed length window of the "Blackman-Harris" type (scaling factor \num{0.42}, equivalent noise bandwidth \num{1.72}\grlinetagger{133})\cite{des08}. After that it removes peaks around burst by the use of a burst mask, extracts the remaining peaks, updates the mask to match the new burst and passes the samples along with the tag to block output.

The \textbf{\texttt{tagged\_burst\_to\_pdu}} block reads the tags and uses them to pack the raw samples belonging to a burst with the metadata from the tag into a standardised message format (a PDU), which is then passed into block output.

The \textbf{\texttt{burst\_downmix}} operates on PDUs. For each PDU, it shifts the centre frequency by the rough estimate of CFO\grlinedownmix{804}, applies a low pass filter, and decimates the signal. Then it searches for start of the burst by analysing magnitude of the filtered signal. Then it seatches for the Unique BPSK synchronisation Word in the frame (see section \ref{s_sat_iridium_signals} and fig. \ref{f_sat_iridium_freq_and_frame_structure} for signal structure details) to acquire a fine estimation of CFO\grlinedownmix{520}, which is again used to shift the signal. After that, the signal is filtered and a correlation function is used to find the start of the unique word and to determine the direction of the burst (uplink or downlink)\grlinedownmix{612}. Lastly, it determines the exact frame size, appends the PDU with the new information and passes it to the block output.

The \textbf{\texttt{iridium\_qpsk\_demod}} block demodulates the frame data. It decimates the signal to one sample per symbol, applies a PLL to remove remaining frequency or phase offset\grlinedemod{358}. Then it perform a QPSK demodulation, checks the unique word and passes the modified PDU to the block output.

The \textbf{\texttt{frame\_sorter}} block sorts the PDU by time of arrival (the processing is multi-threaded, time ordering is thus not assured), and the \textbf{\texttt{iridium\_frame\_printer}} block prints the block into console or a file.

\chapter{Experimental demonstration of SoP positioning}
\label{s_exp}
This chapter describes the evaluation of the navigation system, the data used in testing and the navigation system performance. First, the evaluation data are discussed, as they provide a useful insight into the radio environment of Iridium signals. Second, evaluation methods are explained and results presented. Then, the navigation system is compared to the existing SoP systems. Finally, the details of the system performance are discussed and some explanations offered.

\section{Navigation data}
Ten data sets were collected for the purpose of navigation system testing and validation (\textit{validation} data, prefix \texttt{val}), in addition to data which were used during development (\textit{experimental} data, prefix \texttt{exp}). An example of the data is in \autoref{f_exp_data_val02}. All data are present in the digital appendix.


\subsection{Data parameters}
Select parameters of each data set are in \autoref{t_exp_data_overview}. These include the time of the start of the collection (UTC), duration and length (in data points, i.e. processed IRA frames), the number of satellites whose transmitted frames are present in the data set, mean age of the TLEs, and mean signal parameters - the signal level, the level of the ambient noise, and Signal to Noise Ratio (SNR). 

Detailed parameters of the validation data are in \autoref{t_exp_data_param}. Those are presented as minimum, mean, and maximum values across all data sets. In addition to the parameters mentioned above, the table includes the number of received frames per satellite in the data set, visibility time of satellites, the latitude and longitude of the satellite at the time of frame reception, the extent of satellite coverage, which refers to the distance to the furthest satellite in a given geographical direction (North, South, East, West), the count of some frame types in the data sets, the confidence of the demodulator in the demodulated bits, and the measured Doppler shift.

Of particular importance is the TLE age, as the errors in predicted satellite position can reach several kilometres when the TLE set is one day old (see \autoref{s_pos_tle_sgp4}). In the validation data, the mean TLE age is \qty{16}{hours} and ages up to \qty{40}{hours} were observed. This also demonstrates that the TLE publishing is not regular.

The mean satellite visibility time is around \qty{7}{min}, which corresponds to the theoretical visibility time for Iridium from \autoref{t_sat_general_summary}. The maximum visibility time of over an hour is due to one satellite being captured on multiple passes during one measurement.

Interestingly, the maximum Doppler shift of \qty{37.7}{kHz} is larger than the calculated maximum of \qty{36}{kHz} from \autoref{t_pos_max_fd}. This suggest that a limited reception from beyond the horizon is possible. The mean Doppler shift is positive, which suggests the mean direction of the captured frames is biased to the south, as most of the Iridium constellation orbits (from Earth-based observer perspective) from the south to the north. This is easily explained by the measurement site geographic layout (there is a tree and then a hill right to the north).

% data example
\begin{figure}
    \centering
    \includegraphics[width=5in]{img/exp_data_val02}
    \caption[Example data]{Example data (\texttt{val02}), colour-coded for different satellites, with the IRA transmission frequency shown in red}
    \label{f_exp_data_val02}
\end{figure}

% data overview
\begin{table}
    \centering
    \begin{tabular}{lp{0.75in}lllll}
ID    & Start time          & Duration & Length & Sat. & TLE age   & Signal param.\sc{*}       \\\hline
val01 & 2024-05-05 15:11:42 & 01:06:35 & 1766   & 15     & 21:04:31  & -22.3|-83.5|28.9 \\
val02 & 2024-05-05 17:55:00 & 01:48:49 & 3301   & 18     & 27:36:32  & -22.2|-83.2|28.8 \\
val03 & 2024-05-05 19:45:47 & 01:21:15 & 1474   & 8      & 26:12:40  & -21.8|-82.4|28.5 \\
val04 & 2024-05-06 15:20:49 & 00:40:58 & 1017   & 10     & 02:21:15  & -22.5|-82.8|28.3 \\
val05 & 2024-05-06 18:30:19 & 00:55:05 & 1955   & 10     & 05:45:37  & -23.5|-84.6|28.9 \\
val06 & 2024-05-13 12:22:27 & 01:06:39 & 4141   & 18     & 14:11:58  & -25.0|-86.3|29.1 \\
val07 & 2024-05-13 14:45:22 & 01:06:37 & 2404   & 15     & 15:43:39  & -24.4|-85.7|29.0 \\
val08 & 2024-05-13 16:58:01 & 00:39:51 & 2958   & 6      & 18:59:35  & -25.8|-86.7|28.7 \\
val09 & 2024-05-13 18:49:33 & 00:58:20 & 2005   & 14     & 20:35:00  & -24.0|-85.2|29.0 \\
val10 & 2024-05-14 12:43:10 & 01:23:19 & 4420   & 13     & 16:50:59  & -24.3|-85.7|29.2 \\
\multicolumn{7}{l}{\sc{*}Expressed as Signal level (dB) | Noise level (dB) | SNR (dB)}
    \end{tabular}
    \caption{Captured data overview}
    \label{t_exp_data_overview}
\end{table}



\subsection{Data collection}
The validation data were all collected during the first half of May 2024, in four batches on May 5\sc{th}, 6\sc{th}, 13\sc{th} and 14\sc{th}, using the measurement setup described in \autoref{s_des}. The desired duration of a single dataset was \qty{1.5}{hours}, but this was not always possible due to the weather, as the measurement setup is not rainproof. Generally, the validation data were collected from the afternoon to late evening\footnote{Note that the times in \autoref{t_exp_data_overview} are in UTC and local time during measurement is CEST, so 2 hours ahead.}, under clear to thickly clouded sky.

Each data set was trimmed to remove the frequency variations caused by temperature fluctuations in the SDR after startup. The NORAD IDs of the satellites captured within the data were estimated and used to construct the NORAD ID to Iridium ID mapping table (see \autoref{t_exp_tle_iri_table} for the version valid for the validation data). The data were then processed by the navigation system to produce user position estimates. The parameters of the navigation system, mentioned throughout \autoref{s_des}, are summarised in \autoref{t_exp_final_parameters} along with the values used in the validation data processing.


% final TLE:IRI table
\begin{table}
    \centering
    \begin{tabular}{llll}
100: 73  & 120: 38  & 139: 57  & 155: 25 \\
102: 112 & 121: 42  & 140: 39  & 156: 46 \\
103: 103 & 122: 44  & 141: 51  & 157: 6  \\
104: 110 & 123: 48  & 142: 82  & 158: 18 \\
106: 114 & 125: 69  & 143: 43  & 159: 49 \\
107: 115 & 126: 71  & 144: 74  & 160: 90 \\
108: 2   & 128: 78  & 145: 5   & 163: 3  \\
109: 4   & 129: 79  & 146: 107 & 164: 13 \\
110: 9   & 130: 85  & 147: 7   & 165: 23 \\
111: 16  & 131: 87  & 148: 77  & 166: 96 \\
112: 17  & 132: 88  & 149: 30  & 167: 67 \\
113: 24  & 133: 89  & 150: 40  & 168: 68 \\
114: 26  & 134: 92  & 151: 111 & 171: 81 \\
116: 28  & 135: 93  & 152: 22  & 172: 72 \\
117: 29  & 136: 99  & 153: 8   & 173: 65 \\
118: 33  & 137: 104 & 154: 94  & 180: 50 \\
119: 36  & 138: 109 &          &         \\
    \end{tabular}
    \caption[Iridium NORAD ID to IRA ID mapping table]{Iridium NORAD ID to IRA ID mapping table (X:Y, meaning Iridium X is \texttt{sat-id} Y in IRA frames), valid for the measured data}
    \label{t_exp_tle_iri_table}
\end{table}

% final parameters
\begin{table}
    \centering
    \begin{tabular}{l|llll}
Step size & Initial     & Final & Value limit        &                 \\ \hline
Latitude  & \num{100e3} & 1     &                    & m               \\
Longitude & \num{100e3} & 1     &                    & m               \\
Altitude  & 10,         & 1     & \numrange{0}{3000} & m               \\
Offset    & 3000        & 1     &                    & Hz              \\
Drift     & 0.1         & 0.001 &                    & \unit{Hz\per\s} \\ 
\hline \hline
\param{min-curve-length} & 10 frames & & & \\
\param{max-time-gap}     & 60 s      & & & \\
\param{iteration-limit}  & 500       & & & \\
\param{MIN-FRAME-OCCURRENCE} & 10 frames & & & \\
\end{tabular}
    \caption{Parameters of the navigation systems}
    \label{t_exp_final_parameters}
\end{table}


\section{Navigation system performance}
The main parameters of the performance of the navigation system are
\begin{enumerate}
    \item Accuracy or Bias, i.e. the distance from the estimated user position to the actual user position. Both 2D (horizontal) and 3D (accounting for altitude) accuracy is presented, but in further analysis (e.g. accuracy as a function of measurement time) only horizontal accuracy is considered.
    \item Precision or Standard Deviation, i.e. the deviation of the estimated user position from the mean estimate in the horizontal plane only. Three metrics were considered - all of the data, best \qty{95}{\percent} and best \qty{50}{\percent} of the data.
    \item Time To Fix (TTF), i.e. the measurement time required to achieve an estimate. As the system is capable of estimating user position from very few frames, but such estimation have enormous errors, the TTF is defined as the time at which the estimated position accuracy begins to approach the eventual accuracy of the whole data set.
    \item Number of satellites required to produce an estimate. Same concerns apply here as for the TTF.
\end{enumerate}

The user location estimate produced by the navigation system for each data set, along with the 2D and 3D errors, is in \autoref{t_exp_data_results}.


% results
\begin{table}
    \centering
    \begin{tabular}{l|lllll|ll}
ID & Lat. & Lon. & Alt & Offset  & Drift & 2D err. & 3D err.\\ 
 & (°) & (°) & (m) & (Hz) & (\unit{Hz \per\s}) & (m) & (m) \\ \hline
val01 & 50.571 & 13.839 & 222  & 2004 & 0.131 & 2124 & 2124\\
val02 & 50.581 & 13.846 & 300  & 2140 & -0.173 & 1000 & 1004\\
val03 & 50.580 & 13.843 & 0    & 990 & -0.131 & 1063 & 1083\\
val04 & 50.580 & 13.839 & 0    & 774 & 0.256 & 1155 & 1173\\
val05 & 50.568 & 13.859 & 1060 & 908 & -0.085 & 2674 & 2807\\
val06 & 50.578 & 13.835 & 608  & 5004 & 0.127 & 1395 & 1452\\
val07 & 50.580 & 13.839 & 928  & 5412 & 0.013 & 1137 & 1347\\
val08 & 50.577 & 13.819 & 212  & 4941 & -0.217 & 2237 & 2237\\
val09 & 50.562 & 13.837 & 724  & 4105 & -0.311 & 3102 & 3145\\
val10 & 50.571 & 13.858 & 692  & 4421 & 0.022 & 2367 & 2417\\
    \end{tabular}
    \caption{Navigation system estimates}
    \label{t_exp_data_results}
\end{table}
% H. err. je horizontální? Pak Abs. error je "3D" - to je asi zavadenejší značení.
%% opravil jsemm dobrý postřeh, díky

% Zde - ještě - všechna ta měření val01 až val10 byla měřena s už ode mě korigovaným přijímače? Resp. až po tom zkorigování?
%% ano. Rozdíl v odhadnutém offsetu si vysvětluji tak, že u val06-10 bylo hodně teplo

\subsection{Accuracy and precision}
The accuracy of the navigation system (see \autoref{t_exp_accuracy}) was calculated from the mean of all the position estimates, and it is \qty{1669}{m}. The precision of all data is \qty{1611}{m}, of the best \qty{50}{\percent} it is \qty{572}{m}, and of the best \qty{95}{\percent} it is \qty{1467}{m}. The estimates are visualised in \autoref{f_exp_absolute_accuracy}.  The corresponding probability distribution function is in \autoref{f_exp_absolute_accuracy_dist}, where the blue line represents the 2D error relative to the actual position, and the red line represents the 2D error relative to the mean estimate.

All of the estimates are to the south of the actual position - in fact, the mean error in the E-W direction is \qty{144}{m} eastwards (as opposed to the N-S directional error of \qty{1662}{m} southwards). Furthermore, the grouping of the estimates around the mean is more tight in the N-S direction than in the E-W direction (\qty{2090}{m} and \qty{2836}{m}, respectively).


\begin{table}
    \centering
    \begin{tabular}{l|llll}
Data & Mean lat. (°) & Mean lon. (°) & Bias (m) & Std. dev. (m)\\ \hline
All  & & &                           & 1611 \\
50\% & 50.575  & 13.842   & 1669     & 572  \\
95\% & & &                           & 1467 \\
    \end{tabular}
    \caption{Accuracy and precision}
    \label{t_exp_accuracy}
\end{table}
%  % Jen ne Spread! Je to std. deviation? A Error je "bias"?
% Asi pozor na terminologii. Vtextu to ale určitě vysvětlíte :-).
%% opravil jsem. Terminologie je děs

% absolute accuracy
\begin{figure}
    \centering
    \includegraphics[width=5in]{img/exp_absolute_accuracy.png}
    \caption{Estimated user positions from the validation data}
    \label{f_exp_absolute_accuracy}
\end{figure}

% accuracy dist
\begin{figure}
    \centering
    \includegraphics[width=5in]{img/exp_absolute_accuracy_dist.png}
    \caption{Accuracy and precision cumulative distribution function}
    \label{f_exp_absolute_accuracy_dist}
\end{figure}
% Asi nechápu co je modrá, proč a co se tím rozlišuje. Ale snad text napoví více.
%% modrá je vůči skutečné poloze, oranžová je vůči mean poloze


\subsection{Accuracy and measurement time}
The accuracy of the navigation system as a function of measurement time is shown in \autoref{f_exp_accuracy_vs_meas_time}. It was calculated by splitting the validation data into 5-minute chunks and solving the user position using the cumulative data. The blue curve is the mean across the validation data, with error bars representing the minimum and maximum values in the data set. The datasets shorter than one hour were excluded. The orange line is the frame count in time (mean across validation data). The relationship between measurement time and frame count is linear. Based on \autoref{f_exp_accuracy_vs_meas_time}, the time to fix of the navigation system can be said to be \qtyrange{25}{30}{minutes}.


% accuracy vs time
\begin{figure}
    \centering
    \includegraphics[width=5in]{img/exp_accuracy_vs_meas_time.png}
    \caption{Accuracy as a function of measurement time}
    \label{f_exp_accuracy_vs_meas_time}
\end{figure}
% Hezké. Tak k tomu se budu těšit na text, jak to zhodnotíte. 
%% :)


\subsection{Accuracy and number of satellites}
The accuracy of the navigation system as a function of received satellites is shown in \autoref{f_exp_accuracy_vs_num_of_sats}. Similarly to the measurement time function, it was calculated by splitting the validation data into chunks by satellite, in the descending order of received frames, and solving the user position using the cumulative data. The plot is laid out similarly to  \autoref{f_exp_accuracy_vs_meas_time}. The relationship between the number of received satellites time and frame count is not linear, however, which is due to the sorting of the chunks by size. Based on \autoref{f_exp_accuracy_vs_num_of_sats}, it can be said the that the system can reliably function with Doppler curves from as few as two satellites.


% accuracy vs sats
\begin{figure}
    \centering
    \includegraphics[width=5in]{img/exp_accuracy_vs_num_of_sats.png}
    \caption{Accuracy as a function of number of received satellites}
    \label{f_exp_accuracy_vs_num_of_sats}
\end{figure}
% Super grafy a tabulky. Budu se těšit na text k tomu.
%% :)


\section{Discussion of the results}
As seen in \autoref{f_exp_absolute_accuracy}, the distribution of the user position estimates is significantly biased south and slightly west. In fact, the bias roughly follows the direction of the satellite track (see e.g. \autoref{f_des_exp05_sats_extent}). This is a systematic error of the method, the cause of which is not known to the author. 

It may arise due to incorrect timing, as during development, it was shown that an incorrectly set time of start of recording would move the estimate in the direction of the satellite tracks, by approximately \qty{+5}{km} northward for each added second. Another explanation might be incorrectly estimated frequency offset, or combination of both factors. It is important to note that signal flight time is not compensated, which may account for some of the discrepancy.

As mentioned previously, the distribution of the user position estimates is more tight in the N-S direction (approx. parallel to the satellite tracks) than in the E-W direction (approx. perpendicular to the satellite tracks), which is inconsistent with the reviewed literature (e.g. \cite{sop10, sop12}), where the distribution was significantly broader in the direction perpendicular to the satellite track. This may hint at a possible advantage of the curve fitting method, however, the results presented here are neither numerous nor distinctly distributed enough to prove this.

In both the measurement-time and satellite-number accuracy functions, the accuracy generally improves with measurement time or satellite number, respectively. However, the accuracy can also decrease with a step in meas. time or the number of satellites. This may be due to the added data being of lower quality (e.g. in \texttt{val03} there is a gap caused by a drop in communication), or by the added data having a different drift, which the linear estimator cannot compensate.

Comparison of the measurement time and satellite number accuracy functions reveals the system works much better over a small set of complete Doppler curves than over longer, but incomplete data. This is not surprising, as the method chosen was designed to work over curves rather than data points. Theoretical minimum TTF is therefore around \qty{10}{minutes}, as this is the time required for two consecutive satellites to pass over the user (based on the mean satellite visibility time).

The overall accuracy of the system is lowered by several factors. Firstly, the the satellite positions and velocities were calculated by SGP4 propagation from TLE sets, which can have substantial errors. However, this cannot be avoided unless a dedicated satellite tracking system is developed. It seems that the errors in the TLE-SGP4 positions are somewhat random, as the estimates with old TLE sets are more accurate than the satellite position estimates, as suggested by the analysis in \autoref{s_pos_tle_sgp4}.

Secondly, ionospheric and tropospheric delays were neglected, as was the time of flight of the signal. Furthermore, the satellite clock was assumed to be perfectly accurate, which is likely not the case.

Thirdly, the time of reception of the satellite signal may not be accurate, as it was not provided by the SDR but rather by synchronisation with Iridium IBC frames. There were occasional step changes in the estimated recording start time of hundreds of milliseconds throughout the recordings, which may suggest a timing inaccuracy on the part of either the SDR or the Iridium signals themselves. Note that this method of time synchronisation was not found in literature by the author nor it could verified in this work.

Most important source of inaccuracy is thought to be the SDR frequency offset and drift. This can be somewhat compensated e.g. by introducing a temperature-coupled compensator, or by better estimation of drift by the navigation system, as the linear compensation proved somewhat insufficient.



\section{Comparison to existing SoP positioning systems}
To compare the navigation system (NS) against the existing SoP positioning systems (reviewed in \autoref{s_sop}), the two model SoP systems from \autoref{s_sop_comparison} are used: SoP A and SoP B. The SoP A has an accuracy of \qty{200}{m} and time to fix of \qty{10}{min}, whereas the SoP B has an accuracy of \qty{700}{m} and TTF of \qty{2}{min} (see \autoref{t_sop_comparison}). Note that some of the systems used to construct these example systems utilise sensor fusion with e.g. INS or altimeter. The NS has an accuracy of \qty{1669}{m} and TTF of \qty{30}{min}. Precision or accuracy as a function of number of satellites is not known for the SoP positioning systems from the literature, and therefore cannot be compared.

The navigation system developed in this work is less accurate than the SoP systems from the literature. This is partially due to the existing systems generally using more complex error estimation, especially the estimation of the receiver (and in some cases transmitter) frequency drift. Another factor might be the issues with timing, which were not present in the reviewed systems.

The NS generally requires more time to acquire a fix than the reviewed systems, which is an attribute of the curve-fitting calculation method. The TTF could be significantly lowered if more constellations were used, as this would enable simultaneous complete Doppler curves.

Finally, unlike the existing systems, the NS is a full-stack system capable of performing all the required tasks, which introduces many more avenues for errors, such as satellite or channel misidentification, and makes the identification and correction of these errors more difficult due to the complexity of the system.


% mean data
\begin{table}
    \centering
    \begin{tabular}{p{1in}l|llll}
                                            &        & Min     & Mean    & Max     & Unit\\ \hline \hline
Duration                                    &        & 00:39:51 & 01:06:45 & 01:48:49 & hh:mm:ss     \\ 
Data length                                 &        & 1017 & 2544 & 4420 & frames     \\  \hline
\multirow{3}{1in}{Frames per sat.}              &  min   & 7 & 27 & 120 &      \\ 
                                            &  mean  & 102 & 215 & 493 &      \\ 
                                            &  max   & 245 & 481 & 903 &      \\  \hline
Detected sat.                         &        & 6 & 13 & 18 &      \\  \hline
\multirow{3}{1in}{Satellite visibility  time}   &  min   & 00:00:51 & 00:01:49 & 00:03:39 &      \\ 
                                            &  mean  & 00:05:13 & 00:07:43 & 00:17:42 &      \\ 
                                            &  max   & 00:08:57 & 00:19:21 & 01:46:34 &      \\  \hline
\multirow{3}{1in}{TLE age}                      &  min   & 01:00:14 & 10:07:32 & 14:46:07 &      \\ 
                                            &  mean  & 02:21:15 & 16:56:11 & 27:36:32 &      \\ 
                                            &  max   & 03:22:10 & 24:23:58 & 40:09:30 &      \\  \hline
\multirow{3}{1in}{Satellite latitude}           &  min   & 28.84 & 29.20 & 30.31 & °     \\ 
                                            &  mean  & 46.72 & 49.21 & 52.53 &  °    \\ 
                                            &  max   & 17.38 & 29.22 & 33.00 & °     \\  \hline
\multirow{3}{1in}{Satellite longitude}      &  min   & -20.33 & -12.76 & -4.16 & °     \\ 
                                            &  mean  & 6.64 & 10.19 & 13.32 &  °    \\ 
                                            &  max   & 63.80 & 64.48 & 66.72 & °     \\  \hline
\multirow{4}{1in}{Extent of satellite coverage} &  N     & 1471 & 1547 & 1796 & km     \\ 
                                            &  S     & 2252 & 2375 & 2414 & km     \\ 
                                            &  E     & 250  & 1086 & 1352  & km     \\ 
                                            &  W     & 1271 & 1873 & 2398 & km     \\  \hline
\multirow{3}{1in}{Frame count}                  &  IRA   & 1018 & 2599 & 4443 &     frames \\ 
                                            &  IBC   & 1018 & 2599 & 4443 &     frames \\ 
                                            &  All   & 43013 & 84489 & 159055 & frames     \\  \hline
Demodulator confidence                      &  mean  & 96.7 & 97.0 & 97.3 &  \%    \\ \hline
\multirow{3}{1in}{Doppler shift}            &  min   & -35.15 & -32.44 & -29.52 & kHz \\
                                            &  mean  & -3.54 & 3.83 & 9.08 &      kHz \\
                                            &  max   & 34.10 & 36.55 & 39.73 &    kHz \\
    \end{tabular}
    \caption{Data parameters (minimum, mean, maximum values)}
    \label{t_exp_data_param}
\end{table}

\chapter{Conclusion}
\label{s_con}


\appendix
\chapter{List of Abbreviations}


\begin{itemize}
\setlength\itemsep{0em}
\item[API      ] Application Programming Interface          
\item[CDMA     ] Code Division Multiple Access              
\item[CEP      ] Circular Error Probable                    
\item[CEST     ] Central European Summer Time               
\item[CFO      ] Carrier Frequency Offset                          
\item[ECEF     ] Earth Centred Earth Fixed                  
\item[ECI      ] Earth Centred Inertial                     
\item[EIRP     ] Effective Isotropic Radiated Power         
\item[EKF      ] Extended Kalman Filter                     
\item[FDMA     ] Frequency Division Multiple Access         
\item[FFT      ] Fast Fourrier Transform                    
\item[FLL      ] Frequency Lock Loop                        
\item[GNSS     ] Global Navigation Satellite Systems        
\item[GPS      ] Global Positioning System                  
\item[GSO      ] Geosynchronous Earth Orbit                 
\item[IMU      ] Inertial Measurement Unit                  
\item[INS      ] Inertial Navigation System                 
\item[IQ sample] Complex sample or Quadrature sample        
\item[IRA      ] Iridium Ring Alert                         
\item[ITRS     ] International Terrestrial Reference System 
\item[LEO      ] Low Earth Orbit                            
\item[LNA      ] Low Noise Amplifier                        
\item[MEO      ] Middle Earth Orbit                         
\item[NAVSAT   ] Navy Navigation Satellite Systém           
\item[NCO      ] Numerically Controlled Oscillator       
\item[NORAD    ] North American Aerospace Defense Command
\item[OG1      ] Orbcomm Generation 1                    
\item[PDU      ] Protocol Data Unit                      
\item[PLL      ] Phase Lock Loop                         
\item[PSD      ] Power Spectral Density                  
\item[QAM      ] Quadrature Amplitude Modulation            
\item[QPSK     ] Quadrature Phase Shift Keying           
\item[RF       ] Radio Frequency                         
\item[RMSE     ] Root Mean Square Error                  
\item[RTK      ] Real Time Kinematics                    
\item[RX       ] Receiver, Receiving                     
\item[SDMA     ] Space Division Multiple Access          
\item[SDR      ] Software Defined Radio                  
\item[SGP4     ] Simplified General Perturbations 4      
\item[SK       ] Binary Phase Shift Keying                  
\item[SNR      ] Signal to Noise Ratio                   
\item[SW       ] Software                                
\item[SoP      ] Signals of Opportunity                     
\item[TDD      ] Time Division Duplex                    
\item[TDMA     ] Time Division Multiple Access           
\item[TEME     ] True Equator Mean Equinox               
\item[TLE      ] Two Line Elements                       
\item[TTF      ] Time To Fix                             
\item[TX       ] Transmitter, Transmitting               
\item[UAV      ] Unmanned Aerial Vehicle                 
\item[UHF      ] Ultra High Frequency                    
\item[UTC      ] Universal Coordinated Time              
\item[VHF      ] Very High Frequency                     
\item[VOR      ] VHF Omnidirectional Radio Range         
\item[WGS84    ] World Geodetic System 1984                       
\end{itemize}

\chapter{Data and Code}
The data collected as part of this thesis and the source code of the navigation system constitute the digital appendix to the thesis. The files were submitted along with the digital thesis document (16 zip files packed separately, each containing one dataset, and one zip file with the source code) and the files are available on GitHub (two zip files, one with experimental data and one with validation data, and one zip file with the source code).

The source code includes a \textit{readme} file with installation and usage instructions.

On GitHub, the data and the source code can be downloaded from the Release page of the personal repository of the author:

\begin{center}
   \includegraphics[]{img/qr_code_github}
   \url{https://github.com/voskavoj/DP/releases/tag/submission}
\end{center}


The data are in several folders, each of which contains the following files:
\begin{itemize}
    \item \texttt{output.bits} - these are the demodulated frames
    \item \texttt{decoded.txt} - these are the decoded frames 
    \item \texttt{Iridium.txt} - these are the TLEs for Iridium valid at the time of recording
    \item \texttt{log.txt} - this is the log from the data capture
    \item \texttt{saved\_nav\_data.pickle} - these are the processed data ready for navigation calculations
    \item \texttt{test\_nav\_data.pickle} - these are simulated data used for algorithm tuning (not always present)
    \item \texttt{start\_time.txt} - this file includes the timestamp of the recording start
\end{itemize}

\chapter{List of Free and Open Source software used}
Below is a list of Free and Open Source software used in the navigation system developed in this work, with a link to the respective project's website and a comment about its purpose within the navigation system. The author would like to thank the authors of all of the packages, and the Open Source community as a whole, for keeping their project free of charge and available. The list is sorted alphabetically.

\begin{itemize}
\setlength\itemsep{0em}
    \item Basemap (\url{https://matplotlib.org/basemap/stable/api/basemap_api.html}) for plotting
    \item GNURadio (\url{https://www.gnuradio.org/}) for signal capture
    \item Python (\url{https://www.python.org/}) and the Python Standard Library as the primary programming language
    \item SoapySDR (\url{https://github.com/pothosware/SoapySDR/wiki}) for PlutoSDR support
    \item astropy (\url{https://www.astropy.org}) for frames of reference transformations and operations with time
    \item geopy (\url{https://geopy.readthedocs.io/en/stable/}) for latitude and longitude distance calculations
    \item gr-iridium (\url{https://github.com/muccc/gr-iridium}) for Iridium signal capture and demodulation
    \item iridium-toolkit (\url{https://github.com/muccc/iridium-toolkit}) for Iridium frames decoding
    \item matplotlib (\url{https://matplotlib.org/}) for plotting
    \item numpy (\url{https://numpy.org/}) for vector mathematics
    \item requests (\url{https://docs.python-requests.org/en/latest/index.html}) for downloading TLEs
    \item serial (\url{https://pyserial.readthedocs.io/en/latest/pyserial.html}) for communicating with PlutoSDR
    \item sgp4 (\url{https://pypi.org/project/sgp4/}) for SGP4 predictions
\end{itemize}

\bibliographystyle{unsrt}
\bibliography{references}

\end{document}
