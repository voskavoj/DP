\chapter{Conclusion}
\label{s_con}
\section{Summary}
This thesis studied and demonstrated the Signals of Opportunity (SoP) positioning using the Doppler-shift method and LEO satellites, which is a way of determining user position by evaluating the Doppler shift of satellite signals not intended for navigation. The advantages of SoP positioning were high strength of the signals and their diversity in source, direction, frequency, timing and control. These strengths also presented several challenges, namely the unknown signal transmission time, less accurate and stable transmitter clocks, the inability to synchronise the receiver and transmitter time, as well as a general lack of documentation.
% "experimentally demonstrated" ... divně to zní :-)

The SoP positioning systems reviewed in this work were based on the Doppler shift method, which is discussed in \autoref{s_pos}. In \autoref{s_sat}, the parameters and signals of some LEO satellite systems - namely Iridium NEXT, Orbcomm and Globalstar - were described. The constellations were made of \numrange{30}{66} active satellites transmitting QPSK-modulated signals in the L, VHF and S band, respectively. A simulation confirmed that at least one satellite from each constellation was visible at almost all times.

Some of the current state-of-the-art SoP navigation systems were surveyed in \autoref{s_sop}. Generally, the systems relied on purpose built receivers coupled with an EKF. Most of the systems were based on using the Iridium NEXT, with some using a combination of constellations. Both static and dynamic applications were reviewed. Signal fusion with INS, altimeter or magnetometer data was common. The accuracy of position determination was mostly in the order of \qtyrange{e2}{e3}{\m}, with a measurement time of \qty{30}{s} to \qty{30}{min}. In one case, the SoP positioning system provided a solution where GPS did not (due to foliage).  In terms of accuracy, all of the reviewed systems were outclassed by GPS and other purpose-built navigation systems, however, the accuracy may have already sufficed for some applications. Additionally, some of the systems were as fast as a cold-start GPS. While the reviewed SoP systems did not compare
evenly to GNSS, their performance highlighted the potential of SoP positioning.

The SoP positioning was demonstrated in this work by building a full-stack SoP positioning system, capable of capturing signals from Iridium NEXT satellites, identifying the transmitting satellite and calculating its position, estimating an initial location without any external input, and calculating the user position. The system, described in \autoref{s_des}, was written in Python in a modular fashion, designed to be independent of the actual satellite system used to navigate, and to be used with generic SDR and antenna setup. Since it was a demonstrator, modularity and readability were key design principles, as this enables further research and expansion.

In \autoref{s_exp}, the positioning system was shown to have a 2D accuracy of \qty{1669}{m} and a precision of \qty{1467}{m} (best \qty{95}{\percent} of data). The time to achieve a sufficiently accurate position fix was \qtyrange{25}{30}{minutes}, but a time of \qty{10}{minutes} was possible. The system could reliably function with Doppler curves from as few as two satellites. 

The navigation system developed in this work was less accurate than the SoP systems from the literature. This was partially due to the existing systems generally using more complex error estimation, especially the estimation of the receiver (and in some cases transmitter) frequency drift. The navigation system generally required more time to acquire a fix than the reviewed systems.
% Skutečně je horší? A ony byly nějaké srovnatelné? Vždyť to většinou byly nesrovnatelné - s fůzí s INS a pod. A v čem bylo jejich měření frekvence přesnější? Nemyslím. I oni to museli nakone cměřir podobným principem -PLL/FLL.

\section{Future work}
In the future, the capabilities of the navigation system can be extended, and the system was designed to accommodate such extensions. A custom Iridium receiver can be built, perhaps one optimised only for navigation, which discards or delegates signal demodulation and decoding in favour of more accurate frequency measurements, and which better compensates for offset and drift. The navigation system can also be extended to support additional constellations, such as Orbcomm or Globalstar or even Starlink. Other methods of the user position estimation may be developed, such as the EKF-based observation model method.

The major error sources in the navigation systems are the receiver and satellite frequency drift, the TLE-SGP4 prediction inaccuracies and the issues with determining exact frame reception time. Lesser error sources are the lack of compensation for signal travel time and ionospheric delays. These may also be addressed in the future, for example by developing a better offset and drift compensation or by using multi-frequency measurements to compensate for ionospheric delays. A custom receiver may also alleviate some of these issues.

The main contribution of the designed positioning system demonstrator is that it deals with all the steps of SoP positioning, highlighting the challenges which are not apparent from the literature, and providing a complete framework for further development. More broadly, this work provides a study of the process of SoP positioning and a survey of the current state-of-the-art. Hopefully, it will aid in future research of alternative positioning methods, which may address some of the many challenges of satellite navigation.
% Jako možnosti další práce bych byl trochu nápaditější:
% např. další systémy. A co je hlavním zdrojem chyb? - navrhnout jak je odstranit. Např. lepší zdroj efemerid, měření na více frekvencích (to většinou funguje u dálkoěmrných, ale u Tranzitu psali že i tam vliv ionsféry snad je a zkoumal se)? 

% Kapitola Závěr je taková trochu "těžkopádná". Ale asi ok.