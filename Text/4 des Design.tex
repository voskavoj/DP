\chapter{Design}
\newcommand{\param}[1]{\texttt{#1}}

% todo intro

The navigation system works in several steps, each of which is discussed in detail below:
\begin{enumerate}
    \item Signals from satellites are captured,
    \item data frames are extracted from the signal and decoded,
    \item decoded frames are processed - transmitting satellite is identified, the satellite position at the transmission time is calculated and frequency curves for individual satellites are found,
    \item initial location is estimated, and user position is calculated.  
\end{enumerate}

\paragraph{A note on nomenclature:} Since this chapter discusses the design of the navigation system in detail, an explanation of the used nomenclature should be given. Within the context of this chapter and the navigation system, the \textit{navigation system} refers to the application developed in this work and a \textit{frame} is a data unit sent over a channel by a communication satellite, on the reception of which the navigation system is based. \textit{Extracting} frames means acquiring information about a transmitted frame that do not go beyond its RF properties, such as centre frequency, signal strength or time of arrival. \textit{Demodulating} means converting captured samples into data bits, while \textit{decoding} means converting the bits into meaningful (and human-readable data). A frame has some properties, namely \textit{reception time} $t$, a \textit{centre frequency} $f$ which is shifted from a \textit{base frequency} of the channel $f_B$ on which the frame was transmitted by Doppler shift $f_D$ ($f_D = f - f_B$). Any value with the subscript $U$ refers to the user (i.e. the position of the antenna of the navigation system), a subscript $S$ or prefix \texttt{sat-} refers to the transmitting satellite. Unless otherwise noted, all positions denoted by $x, y, z$ are in ECEF frame (namely ITRS, the International Terrestrial Reference System) and all positions denoted by $\lambda, \phi, h$ are in the geodetic frame ($\phi, \lambda, h$ = latitude (rad), longitude (rad), height (m). In algorithms or conditions, subscript $i$ refers to the $i$-th received frame. References to \textit{Iridium}, unless noted otherwise, always mean \textit{Iridium NEXT}.


\section{Capturing signals}
The first step in determining the user position is to capture signals transmitted by the satellites used for navigation. To achieve that, an antenna and a radio are needed.

\subsection{Antenna setup}
As discussed in section \ref{s_sat_iridium_signals}, Iridium NEXT transmits in the \qtyrange{1626}{1626.5}{\MHz} frequency range, which lies within the L-band. The signal strength on Earth is expected to be in the \qtyrange{-90}{-110}{dBm} range\cite{sop01}, which is in the order of \unit{pW} in absolute signal power.

Thanks to the relative proximity (in frequency terms) of the Iridium signals to those of GPS, a GPS antenna can be used. However, some active antennas, such as the NovAtel GPS-702\cite{des03}, are severely attenuated in the Iridium signal band, and thus cannot be used.

The antenna used in this work is NovAtel GPS-704-X passive GNSS antenna. 
The 3 dB pass band of this antenna is \qtyrange{1.15}{1.65}{GHz} with minimum gain at zenith of \qty{+6.0}{dBic} in the L1 band. Additionally, the antenna features NovAtel’s patented Pinwheel technology for multipath rejection and phase centre stability\cite{des04}. The antenna was chosen due to its suitable characteristics as well as its availability.

To acquire Iridium signals of sufficient strength, a low noise amplifier was found to be necessary. The amplifier used in this work is the Mini-Circuits ZX60-3018G-S coaxial amplifier with frequency range of \qtyrange{0.2}{3}{GHz}, typical maximum output power of \qty{+12.8}{dbm} and gain of \qty{20.60}{dB} for \qty{1671}{MHz}\footnote{Closest value in the datasheet.}\cite{des06}.

\subsection{Radio}
The radio used in the navigation system is the ADALM-PLUTO (in short PlutoSDR) software defined radio development kit. It is a portable self-contained RF-learning module, equipped the AD9363 RF Agile Transceiver. PlutoSDR has one TX and one RX channel with a RF range of \qtyrange{0.325}{3.8}{GHz}. The maximum bandwidth is \qty{20}{Mhz}. The SDR is powered and communicates through USB2.0 with drivers available in Windows and application programming interface (API) in, among others, C, C++ and Python. The APIs are implemented in several development platforms, notably Matlab and GNURadio (via SoapySDR). The reference clock of the SDR has a nominal frequency of \qty{40}{Mhz} and an accuracy of \num{\pm25e-6}\cite{des05}. The block diagram of the SDR is in figure \ref{f_des_pluto_block}.

\begin{figure}
    \centering
    \includegraphics[width=\linewidth]{img/des_pluto_block.png}
    \caption{PlutoSDR block diagram\cite{des05}}
    \label{f_des_pluto_block}
\end{figure}

\subsubsection{Frequency offset and drift}

The synthesizer blocks that generate all data clocks, sample clocks, and local oscillators inside the PlutoSDR transceiver are supplied by a reference clock, which is external to the transceiver, but it is integrated in PlutoSDR and connected to pin \texttt{XTALN} of the transceiver (see fig. \ref{f_des_transciever_diag})\cite{des07}. Due to frequency conversion, any error in the frequency of the reference clock will be multiplied in actual measurements. Given the reference clock nominal frequency of \qty{40}{Mhz} is approximately 40x times lower than the measured band (\qty{\approx1600}{Mhz}), \qty{1}{Hz} offset in reference clock frequency will appear as \qty{-40}{Hz} shift in measured frequency.

\begin{figure}
    \centering
    \includegraphics[width=0.75\linewidth]{img/des_transciever_diag.PNG}
    \caption{AD9363 functional diagram\cite{des07}}
    \label{f_des_transciever_diag}
\end{figure}

Measurements of satellite signals showed that the reference clock of the PlutoSDR suffers from significant frequency offset, which varies with time. As a result, the measured signal frequency differs from actual received frequency by up to \qty{20}{kHz}. A significant cause of the change in the reference clock frequency are variations in temperature, both as a result of internal heating of the SDR components and outer environment changes, which cannot be avoided when measuring outdoors or over a long period of time.

An experiment was carried out, in which the PlutoSDR reference clock frequency was compared to a precise frequency generator (an atomic clock at the FEE CUT). After SDR startup, the reference clock frequency was \qty{39999729}{Hz} (nominal frequency  $f_N$ is \qty{40000000}{Hz} - a difference of \qty{-271}{Hz} or $\num{-6.8e-6} \times f_N$). After 20 minutes of operation, the reference clock frequency was  \qty{39999571}{Hz} (\qty{-429}{Hz} or $\num{-10.7e-6} \times f_N$). After this time, the frequency varied by \qty{\pm 50}{Hz}, but mostly stabilised. Any change in external environment, such as opening the laboratory window, caused significant variation in reference clock frequency.

As a result of this experiment, the PlutoSDR was covered in polystyrene case when measurements were taken, to somewhat stabilise its temperature. Additionally, the SDR was calibrated to account for some of the discrepancy in reference clock frequency, by setting the device to assume the reference clock frequency is \qty{39999571}{Hz}.

This experiment, as well as the satellite signal measurements shows the navigation system needs to account for frequency offset (henceforth referred to as just \textit{offset}) and the change of this offset (henceforth referred to as \textit{drift}) in its calculations. Without precise external frequency source, such as atomic clock (which is unfeasible) or a GPS-synchronised oscillator (which introduces a reliance on GNSS), precise measurements of absolute frequency are not possible. Using a better SDR with more precise or better temperature-compensated oscillator might help, but any system needs to handle unreliable frequency measurements by itself.

\subsection{Measurement setup}
For successful signal capture, the antenna needs to have a wide and mostly unobstructed field of view. Furthermore, the radio should be appropriately protected from rapid changes in environment temperature (such as wind), and adequate power supply must be provided to the radio.

An example of the measurement setup used in this work is photographed in figure \ref{f_des_meas_setup}. It consist of the GPS-704-X antenna with the amplifier (1), an antenna stand (an orange planting box with a bit of cardboard on top) (2), the PlutoSDR in polystyrene case (weighed down by nice rocks the author brought from Greece) (3), a \qty{12}{VDC} laboratory power supply (which is older than the author) (4) powering the amplifier, a Lenovo T440p laptop (5) running the measurement software and a power supply (a slightly rotten power outlet whose circuit breaker the author found after 30 minutes of searching in the \textit{basement}) (6). The measurement setup changed very little during the course of the work.

The view of the antenna is unobstructed above approximately \ang{15} of elevation, with the exception of north direction, which is slightly obstructed by a tree. However, as is apparent in the data, the field of view is overall satisfactory.

\begin{figure}
    \centering
    \includegraphics[width=1\linewidth]{img/des_meas_setup}
    \caption{Measurement setup}
    \label{f_des_meas_setup}
\end{figure}


\section{Frame extraction and decoding}
One of the methods of acquiring and tracking a satellite signal has been described in section \ref{s_pos_tracking_satellite}. However, that refers only to acquiring the signal itself and measuring its frequency. It is highly advantageous to also demodulate and decode the data within the signal. Firstly, some frame types are associated with a specific frequency or a range of frequencies, which greatly simplifies determining the frequency on which the frames were transmitted, and thus their Doppler shift. Secondly, some frame types contain satellite identifiers, which are useful for determining which satellite is transmitting. Thirdly, some frame types contain the position of the transmitting satellite, which may be used as an initial location.

Extracting, demodulating and decoding frames from satellite signal is a specialised task, which is beyond the scope of this work. Therefore, two already existing programs were used - \texttt{gr-iridium}\cite{des09} and \texttt{iridium-toolkit}\cite{des10}.

The \texttt{gr-iridium} package is a module for GNURadio, which handles capturing and demodulating frames sent by Iridium satellites. It works with generic SDRs and outputs decoded Iridium frames as text output in console. Provided it is supplied with proper configuration, the package works from command line and is Windows-compatible, with compiled binaries available in Conda package manager. It requires GNURadio 3.10. The principle of operation is described below.
 
The \texttt{iridium-toolkit} is a Python application capable of decoding demodulated Iridium frames captured by \texttt{gr-iridium}. It can work from command line, is Windows-compatible and does not require compilation nor it relies on GNURadio. It can decode the data in several types of Iridium frames by matching against a known message format.

Both of these packages have already been used in research, such as in \cite{sat08} to capture Iridium Ring Alert messages over a long time.

% \paragraph{A note on nomenclature:} In this context, \textit{extracting} frames means acquiring information about a transmitted frame that do not go beyond its RF properties, such as centre frequency, signal strength or time of arrival. \textit{Demodulating} means converting captured samples into data bits, while \textit{decoding} means converting the bits into meaningful (and human-readable data).

\subsubsection{Process used by \texttt{gr-iridium} to capture and decode Iridium frames}
Below is a description of the process the \texttt{gr-iridium} uses to capture and demodulate Iridium frames. Since the documentation of the package is sparse, the information below comes was inferred from code. References to specific sections of code are in footnotes where appropriate\footnote{The source code is available at \url{https://github.com/muccc/gr-iridium/blob/v1.0.0/}}.

The core of the package is a GNURadio flowgraph (see fig. \ref{f_des_gr_iridium_blocks}\footnote{The flowgraph was created by the author, it is not actually present in graphical form in the package}). Within this GNURadio environment, IQ samples are passed along the solid arrows into the blue nodes of the function blocks and Protocol Data Unit (PDU) messages are passed along the dotted arrows into the grey nodes. The process of extracting and demodulating frames is sequential - it starts in the \texttt{Soapy Custom Source} and ends in the \texttt{iridium\_frame\_printer}.

% command to ease citing of code in gr-iridium
%\newcommand{\grline}[2]{\footnote{gr-iridium v1.0.0: lib/#1\_impl.cc: line #2, available at \url{https://github.com/muccc/gr-iridium/blob/v1.0.0/lib/#1\_impl.cc\#L#2}}}
\newcommand{\grline}[2]{\footnote{gr-iridium/lib/#1\_impl.cc, line #2}}
\newcommand{\grlinetagger}[1]{\grline{iridium\_burst\_tagger}{#1}}
\newcommand{\grlinedownmix}[1]{\grline{burst\_downmix}{#1}}
\newcommand{\grlinedemod}[1]{\grline{iridium\_qpsk\_demod}{#1}}

\begin{figure}
    \centering
    \includegraphics[width=1\linewidth]{img/des_gr_iridium_blocks}
    \caption{\texttt{gr-iridium} flowgraph}
    \label{f_des_gr_iridium_blocks}
\end{figure}

The \textbf{\texttt{Soapy Custom Source}} block handles communication with the PlutoSDR. It sends out raw IQ samples for further processing.

The \textbf{\texttt{iridium\_burst\_tagger}} block identifies Iridium bursts in the captured data and "tags" them by creating metadata with references to burst start time (in terms of samples), magnitude, internal ID, centre frequency, rough carrier frequency offset (CFO) etc. First, the block squares the received signals twice\grlinetagger{527}. Then it performs FFT on the received samples with as fixed length window of the "Blackman-Harris" type (scaling factor \num{0.42}, equivalent noise bandwidth \num{1.72}\grlinetagger{133})\cite{des08}. After that it removes peaks around burst by the use of a burst mask, extracts the remaining peaks, updates the mask to match the new burst and passes the samples along with the tag to block output.

The \textbf{\texttt{tagged\_burst\_to\_pdu}} block reads the tags and uses them to pack the raw samples belonging to a burst with the metadata from the tag into a standardised message format (a PDU), which is then passed into block output.

The \textbf{\texttt{burst\_downmix}} operates on PDUs. For each PDU, it shifts the centre frequency by the rough estimate of CFO\grlinedownmix{804}, applies a low pass filter, and decimates the signal. Then it searches for start of the burst by analysing magnitude of the filtered signal. Then it searches for the unique BPSK synchronisation word in the frame (see section \ref{s_sat_iridium_signals} and fig. \ref{f_sat_iridium_freq_and_frame_structure} for signal structure details) to acquire a fine estimation of CFO\grlinedownmix{520}, which is again used to shift the signal. After that, the signal is filtered and a correlation function is used to find the start of the unique word and to determine the direction of the burst (uplink or downlink)\grlinedownmix{612}. Lastly, it determines the exact frame size, appends the PDU with the new information and passes it to the block output.

The \textbf{\texttt{iridium\_qpsk\_demod}} block demodulates the frame data. It decimates the signal to one sample per symbol, applies a PLL to remove remaining frequency or phase offset\grlinedemod{358}. Then it perform a QPSK demodulation, checks the unique word and passes the modified PDU to the block output.

The \textbf{\texttt{frame\_sorter}} block sorts the PDU by time of arrival (the processing is multi-threaded, time ordering is thus not assured), and the \textbf{\texttt{iridium\_frame\_printer}} block prints the block into console or a file.



\section{Data processing}
After the capture, demodulation and decoding of Iridium frames, some processing is needed to create data suitable for navigation calculation.


\subsection{Parsing frames}
Firstly, since the output of the \texttt{iridium-toolkit} is text, the output needs to be parsed. The output from \texttt{iridium-toolkit} if of the form (1) frame type (e.g. IRA = Iridium Ring Alert); (2) Unix timestamp of the start of the recording\footnote{In the setup in this work, this timestamp is incorrect}; (3) seconds from the start of the recording; (4) centre frequency of the frame; (5) demodulator confidence in percent;  (6) noise values for the frame; (7) frame data length; (8) direction of the frame (UP = uplink, DL = downlink); (9) frame-type-specific data. An example of a received frame is below.

\vspace{\parskip}
\noindent\begin{minipage}{\textwidth}
\begin{verbatim}
    1    2             3              4           5    6
    IRA: p-289693-e000 000025108.8066 1626314368  91%  -29.47|
                  7   8  9
    -090.82|30.79 107 DL sat:067 beam:27 xyz=(+1390,+0092, ...
\end{verbatim}
\end{minipage}
\vspace{\parskip}

The frame-type-specific data for some frame types, namely IRA (Iridium Ring Alert, see section \ref{s_sat_iridium_signals} for details) and IBC (a frame of the duplex channels), contain a satellite identifier, e.g. \texttt{sat:067}. However, this is a system internal ID - it is not public and has no relation whatsoever to the IDs published by NORAD. Additionally, IRA frames contain information about the satellite position in geodetic frame (\texttt{pos} in degrees latitude/longitude, \texttt{alt} in km), e.g.

\begin{verbatim}
    pos=(+38.82/+003.79) alt=797
\end{verbatim}

If the \texttt{alt} is high (generally above \qty{700}{km}), the position refers to the actual satellite, if it is low, the position refers to the location on Earth where the satellite beam is hitting the surface. Some nonsensical values were also observed and filtered out by only accepting frames with $\qty{700}{km} < \texttt{alt} < \qty{900}{km}$ where satellite position information was extracted. The latitude and longitude is given with two significant digits of precision, which, for the latitude of Prague corresponds to accuracy of \qtylist{1.11; 0.71}{km} respectively - not suitable for precise navigation. The process by which Iridium satellites calculate this position is not public, but it can be assumed it has similar errors as the SGP4 algorithm.

The final navigation system is capable of working with IRA and IBC frames. From these, the timestamps (2 and 3), frequency (4) and satellite ID (part of 9) are extracted for further usage where navigation is concerned, and also satellite position (part of 9) is extracted from IRA frames in other applications.

\subsection{Determining frame transmission frequency}
To calculate the Doppler shift of the received frames, it is necessary to identify the channel - and thus the channel base frequency -  on which the frame was transmitted. For IRA frames, this is simple - the transmission frequency of Ring Alert frames is set at \qty{1626.2708}{MHz}. 

For IBC (and other duplex channels), the transmitting channel can be identified by looking which of the known channel centre frequencies is the closest (in frequency space). The channel spacing, which is higher than the maximum Doppler shift for line-of-sight communication, assures that there will be one and only one fit. However, this approach proved not reliable, as higher Doppler shifts were encountered (even in simplex channels where identification is certain).

An additional identification mechanism was thus implemented, where the received frames were sorted by the satellite ID, and a the channel was selected by the above method while observing that the relative Doppler shift to the base frequency among the frames received from a single satellite should not change rapidly:

\begin{equation*}
    \frac{f_{D_i}}{f_{B_i}} \approx \frac{f_{D_{i-1}}}{f_{B_{i-1}}} 
\end{equation*}

However, this still did not prove sufficient and the incorrectly classified channels were a source of significant distortion. Therefore, while the capability remained in the system, the system uses only IRA frames for navigation.

To filer out frames received with erroneous frequency, a thresholding filter was applied ($f_T$ is the threshold of Doppler shift, equal to the channel spacing of $\approx\qty{41.7}{kHz}$):

\begin{equation*}
    f_{D_i} < f_T
\end{equation*}


\subsection{Identifying the transmitting satellite}
The satellite ID with an IRA or an IBC message identifies a satellite within the Iridium ecosystem. However, the rules by which an ID is assigned to a specific satellite are neither public nor apparent. Put simply, the ID within an Iridium frame has no relation to any other public ID of a satellite, such as those assigned by NORAD and used in TLEs. Additionally, as experiments have shown (see below), while NORAD IDs are constant in time, Iridium IDs are not. Therefore, to determine which Iridium frame ID corresponds to which NORAD ID, a mapping table needs to be constructed.

The creation of the mapping table (Iridium frame ID to NORAD ID) in this work takes advantage of the satellite position information contained within an IRA frame. This can be compared to the positions of all active Iridium satellites predicted by the SGP4 algorithm for the same time. The satellite with the closest predicted position to the position in the frame can be assumed to be the one that transmitted the frame. The algorithm (in pseudocode without vector optimisation) is in alg. \ref{a_des_satellite_id}, an example of the table is in tab. \ref{t_des_sat08_table}.

\begin{algorithm}
    \centering
    \begin{verbatim}
satellites = download TLEs for Iridium
frames = load decoded IRA frames
saved_distances = list
frame_ID = table

for frame in frames:
    parse (RX_time, frame_ID, frame_position) from frame

    for sat_ID in satellites:
        predicted_position = run SGP4 for sat_ID at RX_time
        distance = predicted_position - frame_position
        save distance and sat_ID into saved_distances

    closest_ID = find satellite_ID for 
                    minimum distance in saved_distances
    save closest_ID to frame_ID row in saved_IDs

for frame_ID in saved_IDs:
    most_common_ID = find most common ID in frame_ID row
    Iridium ID frame_ID is NORAD ID most_common_ID
    \end{verbatim}
    \caption{Iridium satellite identification}
    \label{a_des_satellite_id}
\end{algorithm}


This process is fairly computationally demanding and not completely error-free. However, when done over several hundred frames with a follow-up application of statistical methods, the satellites can be identified fairly certainly. Uncertainly identified satellites were filtered out by maintaining that the number of frames per satellite is larger than \param{MIN-FRAME-OCCURRENCE}.

\subsubsection{Using a large dataset for mapping IDs}
Assuming the frame IDs remain constantly attached to the same satellite, it would be advantageous to use a large dataset spanning several days or weeks to identify all of the satellites within the Iridium constellation. Such dataset can be found in \cite{sat08}. The data come from Doha, Quatar, and are split into two parts. The first part, which was used for this work, spans 30 days between August 12\textsuperscript{th} and September 10\textsuperscript{th}, 2020 and includes \num{2060289} IRA frames. An excerpt from the mapping table is in tab. \ref{t_des_sat08_table}.

However, a comparison of the table created from the large dataset to a table created from data measured as part of this work in March 2024 reveals the IDs are not in fact constant. Thus, an out-of-date, albeit large dataset cannot be used to create the table. Furthermore, some IDs were observed to change even during this work (Jan. to Mar. 2024). Therefore, a mapping table needs to be generated for each new data.

\begin{table}
    \centering
    \begin{tabular}{l|ll|ll|l}
       & \multicolumn{2}{l|}{Most common} & \multicolumn{2}{l|}{2\textsuperscript{nd} most common} & \\
IRA ID & NORAD ID & Count                 & NORAD ID & Count                                       & Confidence \\ \hline
036 & 122 &  8715 & 169 &   414 & 95\%     \\
038 & 113 &  8774 &     &       & 100\%     \\
039 & 148 &  7158 &     &       & 100\%     \\
040 & 153 &  7300 & 124 &   334 & 95\%     \\
... & ... & ... & ... & \\
069 & 136 &  8930 &     &       & 100\%     \\
071 & 167 &  9314 & 170 &   190 & 98\%     \\
072 & 173 &  9110 & 175 &    16 & 100\%     \\
073 & 133 &  8807 & 169 &   340 & 96\%     \\
... & ... & ... & ... & \\
104 & 116 &  8878 &     &       & 100\%     \\
107 & 142 &  7343 & 124 &   182 &  98\%     \\
109 & 130 &  8870 &     &       & 100\%     \\
110 & 114 &  9186 &     &       & 100\%     \\
    \end{tabular}
    \caption{Example of ID mapping table from large dataset}
    \label{t_des_sat08_table}
\end{table}

\begin{table}
    \centering
    \begin{tabular}{l|lll}
       & \multicolumn{3}{c}{NORAD ID}  \\
IRA ID & Aug./Sep. 2020 & Jan. 2024 & Mar. 2024 \\ \hline
046 & 158 & 156 & 156 \\
049 & 163 & 159 & 159 \\
057 & 119 & \textbf{169} & \textbf{139} \\
... & ... & ... & ... \\
073 & 133 & 100 & 100 \\
078 & 107 & \textbf{179} & \textbf{128} \\
079 & 100 & 129 & 129 \\
    \end{tabular}
    \caption{Example of ID mapping table changing in time}
    \label{t_des_sat08_to_now_table}
\end{table}


\subsection{Identifying a satellite without decoding frames}
Not all frame types include satellite position information or a satellite ID. Furthermore, not all satellite constellations transmit readily decodable signals in the first place. Thus, a mechanism for satellite identification without relying on frame decoding is needed.

One such mechanism (see alg. \ref{a_des_satellite_id_gen} is based on selecting satellites based on plausibility of Doppler shift. First, a rough user location needs to be known. Second, the constellation of the satellite needs to be determined. Thirdly, the channel needs to be identified to determine the Doppler shift. Then, the positions of all the satellites within the constellation are predicted using the SGP4 algorithm. Satellites with negative elevation angles, which cannot plausibly be received, are removed. For all the remaining satellites, the expected Doppler shift for a given channel is calculated. The satellite with the closest estimated Doppler shift is then selected. Note that because in this situation there is no way to distinguish frames belonging to one (albeit unknown) satellite, statistical methods cannot be used.

This method requires prior estimate of user location, therefore it cannot be used alone. On the other hand, it requires no demodulation or decoding, and is thus far more generic. It would be suitable for extending the navigation system to other constellations than Iridium, where the IRA-based method would provide the initial position estimate. Since the navigation system uses IRA frames, this method was not implemented.

\begin{algorithm}
    \centering
    \begin{verbatim}
user_position = estimate of user location
satellites = download TLEs for constellation
frame = load a received frame
parse (RX_time, doppler_frequency) from frame

for sat_ID in satellites:
    predicted_position = run SGP4 for sat_ID at RX_time
    elevation = calculate elevation for
                    predicted_position and user_position
    if elevation < 0:
        continue the loop with the next sat_ID
    sat_doppler_freq = calculate doppler shift for
                            predicted_position and user_position
    save sat_doppler_freq and sat_ID

closest_ID = find sat_ID for
        min(abs(sat_doppler_freq - doppler_frequency) in saved data
    \end{verbatim}
    \caption{Generic satellite identification}
    \label{a_des_satellite_id_gen}
\end{algorithm}


\subsection{Predicting satellite position}
The final step in the data processing for navigation is to predict the position and velocity vectors of the satellite that transmitted each frame. To do this, TLEs for the satellite need to be downloaded, a prediction algorithm run and the resulting vectors converted to a proper frame of reference. This process is also commonly used in other parts of the navigation system (e.g. for satellite identification).

Up-to-date TLEs can be downloaded e.g. from CelesTrak\cite{des11}. For this purpose, a script was written which first checks if up-to-date TLE files for a given constellation can be found locally from a previous download, and if not, downloads TLEs from the server. In either case, it parses the TLEs, calculates the age of the TLE and returns a dictionary. The script can be supplied with a path to a directory, in which case it loads the TLEs from there. In case data were collected but TLEs for them are no longer available, such as for the data from \cite{sat08}, a Special Request can be made on CelesTrak. The required TLEs are then sent by email. As these TLEs are packaged in files per-satellite rather than per-constellation, a  script was created which performs the conversion and automatically selects a most up-to-date TLE set for a given time.

For estimation of satellite position, the widely used SGP4 algorithm was selected (see section \ref{s_pos_sgp4} for details). In Python, this algorithm was implemented e.g. in the \texttt{spg4} package\cite{des12}, which was used in this work. If provided with a Julian date or an array thereof, the algorithm returns an array of postion and velocity vectors in the TEME frame (see section \ref{s_pos_frames_of_ref}).

To convert from the TEME frame to ITRS (ECEF), as well as to handle all other frame-of-reference conversions, the package \texttt{astropy}\cite{des13} was used. This package was also used to handle operations with times and dates, such as converting Georgian dates to Julian dates.

Lastly, the data were packed into a two dimensional array for further calculations. The array is arranged so the "rows" (first index) are the data processed from frames, and the "columns" (second index) are

\begin{itemize}
    \item absolute time of frame reception (Unix time i.e. seconds),
    \item centre frequency of frame (Hz),
    \item base frequency of frame channel (Hz),
    \item NORAD satellite ID (-),
    \item (3x) ITRS position vector (x, y, z; m),
    \item (3x) ITRS velocity vector (x, y, z; \unit{\m\per\s}).
\end{itemize}



\section{Calculating user position}
For calculating the user position from the processed data, the Doppler-curve-fitting method was implemented (see \autoref{s_transit_nav_method} for theoretical basis). % todo label is wrong

In principle, the method attempts to find a state vector $\Vec{s}$, so that the error function $F(c_M, c_E)$, where $c_M$ is the measured Doppler curve and $c_E$ is a trial Doppler curve based on $\Vec{s}$, is minimal. Simply, the method tries to estimate a state vector, whose corresponding Doppler curve provides the \textit{best} match for the measured Doppler curve. The form of the vector $\Vec{s}$, the specific error function and the method of creating the trial Doppler curve $c_E$, as well as the implementation of the search are described in detail below.

\subsection{Estimating initial position}
There are several ways to estimate an initial position for the method. In testing, the navigation system was shown to successfully locate the user while the initial position was well over \qty{1000}{km} away from the final position, the estimate may thus be only very rough. On the other hand, different implementations of the method may require more precise estimate.

Firstly, external knowledge may be used, e.g. it the user knows they are in the Czech Republic, a geographic centre of CR may be used.

Secondly, the position of any received satellite may be used, as those can be assumed to be relatively close. As apparent in \autoref{f_des_exp05_sats_extent}, the satellites are generally located within \qty{2500}{km}, with the majority of captured frames coming from less than \qty{1000}{km} away. An average of the satellite positions can be used to obtain a better estimate (see figure). However, the reliability of this method depends on the field of view of the antenna being similar in all directions - if, for instance, the antenna is shielded from the north, the resultant position will be significantly shifted southward.

Thirdly, as described in \autoref{s_transit_nav_method}, the user is (in an ideal case) located on a line perpendicular to the satellite track, intersecting the satellite track at the position where the Doppler shift of the satellite signal is zero ($r_{f_D = 0}$). It follows that the position on the satellite track which is closest to the user is $r_{f_D = 0}$. Obtaining this position is straightforward as long as the measured Doppler curve for a given satellite passes through zero. The received data are split into curves per satellite and per channel, that is, a curve is formed by all observations with constant $f_B$ and satellite ID. Furthermore, curves with longer gap between two frames than \param{MAX-CURVE-GAP} seconds and split. This was necessary for very long captures, where a single satellite is captured on two or three passes. Finally, curves with fewer than \param{MIN-CURVE-LENGTH} frames are discarded. For each curve the point of $f_D = 0$ is interpolated. Positions from all successfully interpolated curves are then averaged to obtain the initial position estimate.



The navigation system in this work uses the third method, with the second method as backup if no curves were interpolated. % todo tohle ted neni pravda, ale je to dobry napad do kodu
For further navigation calculations, the curves are joined together again to form a single curve.


\begin{figure}
    \centering
    \includegraphics[width=1\linewidth]{img/des_exp05_sats_extent.png}
    \caption{Example of tracks of captured satellites}
    \label{f_des_exp05_sats_extent}
\end{figure}


\subsection{Trial Doppler curve generation}

