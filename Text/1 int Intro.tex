\chapter{Introduction}
\label{s_int}
The modern world is increasingly reliant on navigation systems in all areas and industries, from transport through agriculture to security. Be it precision agriculture, geo-fencing, the guidance of aircraft, ships, autonomous vehicles, logistic coordination and tracking, emergency search and rescue, house arrest control, all kinds of military operations, and of course personal navigation, they all share the need for a quick, reliable and accurate positioning.

The need perhaps first arose in military context, and indeed the first navigation systems more sophisticated than a human with a map and a compass or a sextant were developed by the various militaries of the world in the 20\sc{th} century, but the civilian sector soon caught up. Systems such as Loran C, Non Directional Beacon or VHF Omnidirectional Radio Range (VOR) were ground based radio navigation systems. From the middle of the century the access to space allowed for satellite-aided navigation systems, such as Transit, Dora, and then the Global Positioning System (GPS) and Global Navigation Satellite Systems (GNSS) in general.

Today, the GNSS provide the bulk of the navigation services for the applications outlined above. GNSS are commonly based on satellites in Middle Earth Orbit (MEO) around \qty{20000}{km} above the surface, although some systems utilise Geosynchronous Earth Orbits (GSO) too, with constellations of around 30 satellites. The large orbital height allows a relatively small constellation to have several satellites simultaneously visible   from the surface of the Earth. That is required, as the navigation method used by the GNSS, the range method, requires at least four simultaneously visible satellites. However, it carries a significant drawback - the received signal is very weak, which not only requires complex signal processing for acquisition, but also makes the GNSS signals susceptible to jamming and spoofing\footnote{\textit{Jamming} means blocking out a signal with a stronger one, whereas \textit{spoofing} means emulating the signal by another in order to deceive the receiver}.

In the transport context, perhaps more than in any other, the consequences of a spoofing or a jamming attack are significant, ranging from disruption of the flow of goods and people to a possible harm to human lives. The availability of powerful hardware and software for spoofing lowers the barriers potential adversaries face in an attack, whereas the need for the architecture of the GNSS to remain compatible makes adaptation difficult\cite{int01}.

However, there need not be malice involved - simple unavailability of the GNSS, poor signal quality or insufficient coverage may suffice to cause disruption. GNSS are run by governments and government-like entities, who may decide to limit the service e.g. in times of war, the GNSS signals are easily degraded by buildings or dense trees, can be jammed by radar, radios or electromagnetic interference and the coverage is somewhat geographically limited (e.g. GPS performance degrades in the polar regions\cite{int01, int02}).

Solutions were developed to mitigate the vulnerabilities in GNSS. Coupling with an Inertial Navigation System (INS) is one, physical layer authentication\cite{sat08} or crowd-sourcing, i.e. the fusion of information from multiple sources, are some of the other\cite{int01}. However, another solution might be to rely on an independent navigation system, one with much stronger signals and much more diverse signal sources - Signals of Opportunity (SoP) navigation using communication satellites.

The access to space allowed for satellite communication systems to be deployed, and the industry is thriving today. There are multiple constellations, some with high tens of satellites, some with high thousands (see \autoref{s_sat} for details). As a result, the frequency space is filled by communication transmissions, which can be readily used for navigation.

In the context of this work, Signals of Opportunity positioning\footnote{Within the thesis, \textit{positioning} and \textit{navigation} are used somewhat interchangeably. The system developed by this work deals exclusively with a static user, and is thus a \textit{positioning} system, but other system referenced are capable of working in dynamic mode as well and the findings of this thesis are applicable to dynamic modes as well.} is the process of finding the user position by the use of signals which were not designed nor transmitted for the purpose of positioning. In this work, only satellite SoP are considered, although SoP positioning can be performed over signals from any source. 

This thesis studies and experimentally demonstrates the SoP positioning using the Doppler method.



\section{The structure of the thesis}
In the remainder of this chapter, SoP positioning is explained, its advantages and challenges are discussed and some use cases are presented.

In \autoref{s_pos}, the theoretical foundation for the Doppler shift method is laid. Some ways of implementation of the method in the context of SoP positioning are discussed. A process of capturing a satellite signal is described, as is a way of determining a satellite position. Lastly, the coordinate frames of reference relevant in this work are described.

In \autoref{s_sat}, some Low Earth Orbit satellite systems potentially usable for SoP navigation are described - the Iridium NEXT, Orbcomm, Globalstar and Starlink constellations. The constellation, orbital and signal parameters are listed and the signal types are described, with a particular focus on the signals with properties useful for SoP navigation. Lastly, a satellite visibility analysis is conducted for select systems.

The \autoref{s_sop} is a survey of SoP systems in the current research. The methods used, signals exploited and results achieved are described. Additionally, two dedicated navigation systems which use the Doppler shift method are depicted, to provide a basis of comparison. The SoP systems are then compared against GNSS.

The \autoref{s_des} describes in detail the development of a full stack navigation system. The implementation of satellite signal capture, demodulation and decoding is described. Satellite identification system is developed, and the algorithm for user position calculation is devised.

In the \autoref{s_exp}, the developed navigation is tested and its parameters evaluated.

\section{Signals of Opportunity navigation challenges}
Due to the nature of the signals, SoP positioning presents several unique challenges.

Firstly, the signal transmission time is generally not known. This limitation rules out the application of the pseudo-range method of navigation, employed e.g. by GPS.

Secondly, the clocks onboard the satellites transmitting the SoP signals generally do not need to be as accurate as GNSS clocks. This means that the transmission frequency of SoP signals suffers from larger errors due to clock offset and drift. This also precludes using any sort of timestamp within the signal for the pseudo-range method, even if it was present and decoded.

Thirdly, the identity or position of the transmitting satellite are generally not included in the SoP signal message. This places additional requirements on an SoP navigation system and introduces additional errors.

Fourthly, the signals are not guaranteed to be available globally, and can be selectively turned off by the system operator.

Fifthly, the SoP do not generally contain any parameters which can be used to synchronise the receiver clock, correct the satellite orbital model or the ionospheric propagation model, as is the case for e.g. GPS\citep{pos04}{22}.

For these and other reasons, SoP navigation systems generally use the Doppler shift method (see \autoref{s_sop}), which does not rely on message transmission time. The Doppler method (see \autoref{s_pos}) does not rely on the signal reception time either, but the satellite position prediction likely does (see \autoref{s_des} for a discussion about various implementations). As a result, the challenges mean that an SoP navigation system will likely never be as accurate as a GNSS.



\section{Signals of Opportunity navigation advantages and possible use cases}
\label{s_int_use_cases}
The nature of the SoP also provides some significant advantages over GNSS.

Firstly, the signals mainly originate from LEO satellites, orbiting around \qty{800}{km} above the surface (see \autoref{s_sat}), which is around \numrange{20}{30} times closer than the MEO GNSS satellites. This results in the SoP being much stronger than GNSS signals (at least by an order of magnitude, although some sources claim up to four orders of magnitude\cite{sop08}), and therefore much more capable of penetrating tree cover or buildings. Indeed, the capability of obtaining a fix with SoP navigation, but not with GPS was demonstrated in \cite{sop12}.

Secondly, the signals originate from many different satellites, with different orbits and transmission frequencies, which results in frequency and direction-diverse signal set. This can be used to eliminate e.g. ionospheric refraction. It also makes jamming the signals very difficult.

Thirdly, the satellites are owned and operated by many entities, making the overall system more diverse and thus more resilient to outages.

The SoP navigation is not only useful as a standalone system, but it can be readily used to augment other systems, be it GNSS, or by fusion with INS or magnetometer etc.

These traits make the SoP readily usable as a backup or augmentation navigation system, especially in GNSS-denied conditions.  For instance, it can be used as an emergency localisation system in aircraft, improving safety or aiding in emergency search and rescue. A case study may be the loss of Malaysia Flight MH370 and the subsequent search operation, hampered by a lack of location data about the aircraft\cite{sop13}.

Furthermore, the SoP navigation system can aid in navigation reliability and safety. In this form, the SoP system can function as a part of a spoofing-detection system, providing a second source of navigation information. Resistance to spoofing is vital for ships relying heavily on autonomous navigation, as GNSS spoofing is one of the dangers faced by mariners from sea pirates\cite{int03}.

% Perfektní
%% Díky :)