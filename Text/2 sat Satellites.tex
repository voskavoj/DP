\chapter{Overview of LEO satellite systems suitable for SoP positioning}
\label{s_sat}
This chapter describes current LEO satellite constellations which are readily usable for SoP navigation, namely  Iridium NEXT, Orbcommm, Globalstar and Starlink. For each system, the orbital parameters, constellation status, frequency ranges and communication schemes are described, in addition to select signal types deemed of particular importance for this work.

Parameters of the systems are summarised in table \ref{t_sat_general_summary}.

\begin{table}
\caption[Summary of approximate parameters of LEO satellite systems]{Summary of \emph{approximate} parameters of LEO satellite systems: number of active satellites, number of orbital planes and inclinations, altitudes, visibility for one satellite, frequency ranges, signal strengths at ground level and modulation used}
\label{t_sat_general_summary}
\centering
\begin{tabular}{l|llll}
            & Iridium               & Orbcomm             & Globalstar           & Starlink \\ \hline
Satellites  & 66                    & 30                  & 48                   & $>$5000 \\
Orb. planes & 6 $\times$ \ang{86.4} & 4 $\times$ \ang{47} & 8 $\times$ \ang{52}  & $>$180 \\
Altitude    & \qty{781}{\km}        & \qty{715}{\km}      & \qty{1414}{\km}      & \qty{550}{\km} \\
Visibility  & \qty{7}{\min}         & \qty{7}{\min}       & \qty{14}{\min}       & \qty{4}{\min} \\
Frequency   & \qty{1616}{\MHz} -    & \qty{137}{\MHz} -   & \qty{2483.5}{\MHz} - & \qty{10.7}{GHz} -\\
range       & \qty{1626.5}{\MHz}    & \qty{138}{\MHz}     & \qty{2500}{\MHz}     & \qty{12.7}{GHz} \\
Strength    & \qty{-100}{dBm}       & \qty{-118}{dBm}     & not found            & not found \\
Modulation  & DE-QPSK               & SD-QPSK             & QPSK                 & 16QAM \\      
\end{tabular}
\end{table}


\section{Iridium NEXT}
\label{s_sat_iridium}
Iridium satellite constellation, owned and operated by Iridium Communications Inc., provides global communications coverage for satellite phones, pagers and other devices, as well as an interface to terrestrial communications networks via several Gateways.

The system became operational in 1998 and saw a total of 96 satellites launched, before the constellation was completely overhauled between 2017 and 2019, replacing all of the original satellites with Iridium NEXT satellites. Currently\footnote{As of October 2023}, no original satellite remains operational, however, 25 satellites remain defunct on orbit\cite{sat06}. In this work, unless specifically noted, Iridium refers to Iridium NEXT.

The constellation consists of 66\footnote{Satellites 102-114, 116-123, 125-160, 163-168, 171-173, 180. All ranges are inclusive.} satellites, 11 each in six orbital planes, spaced \ang{30} apart. The altitude of the satellites is approx. \qty{781}{\km}, the inclination is \ang{86.4} and the orbital period is approx. \qty{100}{\min}. For a ground user, one satellite is visible for about \qty{7}{\min}. The constellation covers the entire global surface\cite{sat01}.

Communication is carried over Ka-band for satellite-satellite links and satellite-gateway links, and over L-band (\qtyrange{1616}{1626.5}{\mega\hertz}) for satellite-user links. Each satellite provides 48 individual spot beams, sharing 240 traffic channels with a frequency re-use pattern\cite{sat07}. At ground level with an Iridium antenna, the signal strength can be expected to be \qtyrange{-90}{-110}{dBm}\cite{sop01}.

The Iridium NEXT onboard clocks have a stability better than \num{10e-9} and a drift less than \qty{3}{\hertz\per\s}, which is sufficient for time transfers with accuracy in the order of \qty{e0}{\micro\s}\cite{sop11}.
% Opravdu Iridium má drift až 3 Hz/s ? To je dost. Kolik že to vycházelo nám u přijímače?

\subsection{Iridium signals}
\label{s_sat_iridium_signals}
Due to the scope of this work, only the satellite-user (L-band) signals are examined, as the Ka-band signals are either not globally transmitted or not aimed towards the Earth.

Iridium uses a combination of Space Division Multiple Access (SDMA),  Frequency Division Multiple Access (FDMA), Time Division Multiple Access (TDMA) and Time Division Duplex (TDD) multiple access schemes\cite{sop11}. For every spot beam, channels\footnote{A channel is a specific FDMA frequency and TDMA timeslot\cite{sat07}} are implemented using TDMA architecture based on TDD using a time frame\cite{sop12}. The structure of a TDMA frame is illustrated in Fig. \ref{f_sat_iridium_freq_and_frame_structure}.

In the \qtyrange{1626}{1626.5}{\MHz} band Iridium uses, there are 252 carriers with carrier spacing of \qty{41.667}{\kHz}, grouped into 31 sub-bands of 8 and one of 4. There are 5 simplex carriers in the band with a spacing of \qty{35}{\kHz}, one of which is the Ring Alert, and the remaining four are messaging carriers for paging and acquisition. The polarisation of both the uplink and downlink band is right-hand circular. Iridium uses DE-QPSK modulation\cite{sat04}.

\begin{figure}
    \centering
    \includegraphics[width=0.75\linewidth]{img/sat_iridium_freq_and_frame_structure.png}
    \caption{Iridium frequency allocation and frame structure\cite{sop12}}
    \label{f_sat_iridium_freq_and_frame_structure}
\end{figure}

\subsubsection{Paging}
Paging signals enable users to receive ringing and paging messages during heavier atmospheric fading conditions and in buildings where attenuation is greater. There are five paging channels, one for alert (Ring Alert, detailed below) and five for transmitting paging messages to the receive-only terminals.  The transmission duration does not exceed \qty{20.32}{\ms}\cite{sat09}.

\subsubsection{Ring Alert}
The Iridium Ring Alert signal is an unencrypted downlink-only simplex channel with a carrier frequency of \qty{1626.2708}{\MHz}\cite{sat04}. Each of the 48 satellite beams transmits a Ring Alert message every \qty{4.32}{\s}, for a transmission period of \qty{90}{\ms} of one satellite\cite{sat07}, which can be used to keep track of a specific satellite\cite{sop11}.

The transmit power for the Ring Alert channel is higher than that of voice/data channel, so as to enable the mobile earth terminals to receive ring alerts even when their antennas are stowed\cite{sat09}.

The Ring Alert Message contains the following information\cite{sat08}:
\begin{itemize}
    \item Satellite ID - a numeric identifier (\numrange{2}{115}) of the transmitting satellite, which notably does \emph{not} correlate with the satellite NORAD identifier,
    \item Beam ID - a numeric identifier (\numrange{0}{47}) of the transmitting beam,
    \item Latitude - the current \emph{ground} latitude of the satellite in degrees, with two decimal places of precision, calculated by the satellite,
    \item Longitude - same format as Latitude and
    \item Satellite altitude - altitude above the surface in \unit{\km}, which was shown in \cite{sat08} to be unreliable.
\end{itemize}

Importantly, the inclusion of satellite ID in the Ring Alert message, combined with the fixed and known frequency of the Ring Alert channel and the absence of encryption, enables any receiver to quickly determine which satellite is transmitting. The Ring Alert signals are thus very useful for SoP navigation using Iridium.


\section{Orbcomm}
Orbcomm satellite constellation provides private internet access and machine-to-machine communication services primarily to industrial customers, such as transportation companies or manufacturers. The first launch of the system occurred in 1991. There were several generations of satellites, of which two amounted to significant numbers - the first production generation, referred to as Orbcomm Generation 1 (OG1), and the latest one, Orbcomm Generation 2 (OG2), which began launching in 2012. TLEs indicate that 60 satellites are currently\footnote{As of October 2023} in orbit, of which 36\footnote{Satellites 4-15, 18-21, 23, 27, 30, 31, 32 (partially), 34, 35 (OG1); 103, 107-110, 112-118 (OG2). All ranges are inclusive.} are active, 24 of OG1 and 12 of the OG2 configuration\cite{sat12}.

The Orbcomm satellites orbit in four orbital planes inclined at \ang{47} at an altitude of \qty{715}{km}. The orbital period is approx. \qty{99}{\min} and a satellite is visible to a ground observer for about \qty{7}{\min}\cite{sat11, sop08}. % todo actual orbital planes from TLEs

The Orbcomm communication system consist of the Space segment (satellites), Ground segment (Earth-based gateways and communication centres) and the Subscriber segment (Earth-based paying customers)\cite{sop08}. Orbcomm satellites are also equipped to receive Automatic Identification Signals from maritime traffic, however this is not a navigation system, but rather navigation information collection system\cite{sat11}.

\subsection{Orbcomm signals}
Downlink signals from the Orbcomm satellites to the subscriber and ground segments are in the VHF band, within \qtyrange{137}{138}{MHz}. There are 12 channels for communication with the subscriber segment and one for the gateways\cite{sop08}.

Each satellite is assigned one of the 12 channels. Over the constellation, the frequencies are reused in a scheme such that signals of equal frequency never overlap. Frequencies of the channels are listed in table \ref{t_sat_orbcomm_channels}. The bandwith of all channels is \qty{25}{kHz}, except the gateway channels, whose is double\cite{sat13}. Importantly, this assignment provides a means of satellite identification, if a table of assigned frequencies is provided or constructed experimentally.

\begin{table}
\caption{Orbcomm downlink frequency channels\cite{sat13}}
\label{t_sat_orbcomm_channels}
\begin{tabular}{ll}
Channel & Frequency (MHz) \\ \hline
S-1     & 137.2000        \\
S-2     & 137.2250        \\
S-3     & 137.2500        \\
S-4     & 137.4400        \\
S-5     & 137.4600        \\
S-6     & 137.6625        \\
S-7     & 137.6875        \\
S-8     & 137.7125        \\
S-9     & 137.7375        \\
S-10    & 137.8000        \\
S-11    & 137.2875        \\
S-12    & 137.3125        \\
Gateway & 137.5600       
\end{tabular}
\end{table}

The downlink signals are modulated using Symmetric Differential Quadrature Phase Shift Keying (SD-QPSK). The transmission power varies from \qtyrange{10}{40}{W} with effective isotropic radiated power (EIRP) about \qty{12}{dBW}. Ground signal strength is about \qty{-118}{dBm}. The polarisation is right-hand circular\cite{sop08, sat11}.

In addition to the downlink communication signals, Orbcomm satellites are equipped with a \qty{1}{W} unmodulated \qty{400.1}{MHz} (UHF) beacon\cite{sat13}. However, in \cite{sop08} it was shown that this beacon was missing or turned off in the tracked satellites.

\section{Gobalstar}
Globalstar is a satellite communication system designed to provide voice and low-rate data communication to user terminals on the earth. The satellites connect the users and ground-based gateways by retransmitting the user terminal signals\cite{sop07, sat14}.

The constellation had two distinct generations. The first generation launched from 1998 and included 48 LEO satellites, none of which remain operational. The current\footnote{As of October 2023}, second, generation has 24\footnote{Satellites 074-097. All ranges are inclusive.} operational satellites, launched between 2010 and 2013. Neither the orbital configuration nor the signal spectrum has changed between the generations\cite{sat15}.


The satellites orbit at an altitude of \qty{1400}{km} in 8 planes inclined at \ang{52}. The orbital period is \qty{114}{min}. Each Globalstar satellite is visible for around \qty{14}{\min}\cite{sat14, sat15}

The number of simultaneous visible satellite depends on latitude. For the original constellation and a latitude of \ang{50}, a probability of seeing one, two or three satellites simultaneously in a given moment is respectively \qtylist{19;71;10}{\percent}\cite{sop07}. For the second generation, the probabilities can be expected to be much lower.

The Globalstar satellites use a phased array antennas with 91 high power amplifiers, each providing an output power of \qty{4}{W} (for a total of \qty{364}{W})\cite{sat14}. Satellite fractional frequency error standard deviation is specified as \num{e-9}\cite{sop07}.


\section{Globalstar signals}
The Globalstar constellation implements four types of signals - satellite-user downlink (S band), user-satellite uplink (L band), and satellite-gateway downlink and uplink (C band, \qtyrange{5}{7}{GHz}). All transmissions use Code Division Multiple Access (CDMA)\cite{sat10}.

The user downlinks are in the frequency range from \qtyrange{2483.5}{2500}{MHz}. Each satellite transmits in 16 beams, utilising the full allocated range of \qty{16.5}{MHz}. The modulation is QPSK\cite{sat14}.

\section{Starlink}
Of the systems discussed, Starlink has by far the most satellites. However, due to the downlink frequency exceeding \qty{10}{GHz}, it is not feasible to process Starlink signals in this work. Starlink is therefore discussed only briefly for the sake of completeness.

Starlink is an satellite internet and telephony constellation. It currently\footnote{As of October 2023} features over \num{5000} satellites in over 180 orbital planes, inclined mostly at \ang{52} or \ang{97}. Up to 24000 - 42000 satellites are possible\cite{sop02}. The satellites orbit at an altitude of around \qty{550}{km}, with an orbital period of \qty{91}{min} and single satellite visibility time of around \qty{4}{min}.

The starlink downlink signal is in the \qtyrange{10.7}{12.7}{GHz} band, with several channels of a \qty{240}{MHz} bandwidth. Furthermore, several \qty{1}{MHz} bandwidth tones were identified within the band by experimentation. The structure of the signal is not public. The reception of signals within this band usually requires a low-noise block down-converter and a parabolic antenna reflector\cite{sop04}.

\section{Summary}
Iridium NEXT, Orbcommm, Globalstar and Starlink were discussed in this chapter, but only the frequency ranges of Iridium NEXT, Orbcommm, and Globalstar fit within the bounds set by available antennas. Thus, only those three will be considered in this work.

Iridium NEXT constellation has the most satellites, while Globalstar has the most simultaneously visible satellites. At least one Iridium NEXT and Globalstar satellite is always visible, while for Orbcomm this is not the case. As many as 6, 6, and 13 Iridium, Orbcomm and Globalstar satellites respectively are visible simultaneously. On average, 2.2, 1.6, and 5.8 respectively are visible simultaneously. These values are summarised in \autoref{t_sat_simul_vis} and visualised in \autoref{f_sat_simul_vis}. Elevation angles for the visible satellites of one constellation (Iridium NEXT) are in \autoref{f_sat_elev_iri}.

The visibility was simulated relative to the location of CUT FEE (\ang{50;6;11}N, \ang{14;23;32}E, \qty{225}{m}) over 36 hours starting on May 1\textsuperscript{nd} 2024 00:00 UTC. The plots are 2-hour excerpts from the middle of the time window.
% CTU, ne CUT :-)

The signal power at the receiver location for Iridium and Orbcomm is \qtylist{-100;-118}{dBm} respectively, compared to the power of GPS signal of \qty{-128}{dBm}\citep{pos04}{11} (difference of 63x, 10x respectively), which clearly illustrates the advantage of signal strength the SoP positioning systems enjoy over GNSS.
% 28 dB je 630x, ne 63x

\begin{table}
    \centering
    \begin{tabular}{llll}
 & Iridium & Orbcomm & Globalstar \\ \hline
Min & 1 & 0 & 1 \\
Avg & 2.2 & 1.6 & 5.8 \\
Max & 6 & 6 & 13 \\
    \end{tabular}
    \caption{Number of simultaneously visible satellites for Iridium, Orbcomm and Globalstar over 36 hours}
    \label{t_sat_simul_vis}
\end{table}

\begin{figure}
    \centering
    \includegraphics[width=0.9\linewidth]{img/sat_simul_vis}
    \caption{Number of simultaneously visible Iridium, Orbcomm and Globalstar satellites during 2 hours of a day}
    \label{f_sat_simul_vis}
\end{figure}

\begin{figure}
    \centering
    \includegraphics[width=0.9\linewidth]{img/sat_elev_iri}
    \caption{Elevation angles of Iridium satellites during during 2 hours of a day}
    \label{f_sat_elev_iri}
\end{figure}

% OK. Super.
